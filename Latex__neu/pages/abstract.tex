\begin{center} \section*{Abstract} \end{center} 
\newenvironment{keywords}{
	\begin{flushleft}
	\small	
	\textbf{
		\iflanguage{ngerman}{Schlüsselwörter}{\iflanguage{english}{Keywords}{}}
	}
}{\end{flushleft}}

% Deutsche Zusammenfassung
\begin{abstract}\noindent 
Diese Arbeit stellt die Konzeption einer mobilen Lern-App namens \gqq{LearnAhead} vor, die es Benutzern ermöglicht, Lerninhalte zusammenzufassen und basierend darauf Tests zu erstellen. Benutzer können Tests mit einer Vielzahl selbst ersteller Fragen füllen. Die Applikation bietet Funktionen zum Gruppieren von Inhalten in Kategorien und zur Zuweisung von Lernzielen. Basierend auf diesen Lernzielen generiert die Lernapp mithilfe eines Algorithmus einen empfohlenen Lernplan für den Benutzer. \newline

\noindent
Die Anforderungen für die Arbeit umfassen die Möglichkeit, Lerninhalte zusammenzufassen, Tests zu erstellen, Inhalte zu kategorisieren, auf Android-Geräten verfügbar zu sein, Lernziele bereitzustellen, automatisch erstellte Lernpläne zu generieren und Benutzerprofile mit Benutzernamen und Kennwörtern zu haben. \newline

\noindent
Der theoretische Hintergrund beschäftigt sich mit dem Spacing Effect, einer Lerntechnik, die auf den Forschungen von Hermann Ebbinghaus zum Gedächtnis beruht. Der Spacing Effect besagt, dass regelmäßige Lernintervalle mit Pausen dazwischen dazu beitragen, Informationen effektiver zu behalten. Indem der Spacing Effect integriert wird, kann die App automatisch optimale Lernpläne für die Benutzer berechnen und sicherstellen, dass gelerntes Material mit wiederholten Abständen wiederholt wird. \newline

\noindent
Die App enthält eine Tagging-Funktion, um den Lernprozess zu vereinfachen. Benutzer können bestimmte Lerninhalte mit Tags versehen, was Funktionen wie die Erstellung von Tests erleichtert. Das Tagging ermöglicht es Benutzer Informationen effizient zu organisieren und abzurufen. \newline

\noindent
Die App integriert auch Gamification-Elemente, um Motivation und Engagement zu steigern. Gamification im Bildungsbereich beinhaltet die Anwendung von Spielelementen wie Punkten und ein zugehöriges Levelsystem, um ein interaktives Lernerlebnis zu schaffen. Diese Strategie kann die Motivation der Lernenden steigern, die Teilnahme erhöhen und letztendlich zu verbesserten Lernergebnissen führen. \newline

\noindent
LearnAhead implementiert Gamification durch ein Punktesystem, bei dem Benutzer Punkte für verschiedene Aktivitäten wie das Abschließen von Tests, das tägliche Einloggen, das Erreichen von Lernzielen und das korrekte Beantworten von Fragen verdienen. Die Punkte tragen zur Aufwertung des Benutzerprofils bei, bieten sofortiges Feedback und ein Gefühl von Leistung und Fortschritt. \newline

\noindent
Zusammenfassend bietet die App den Studierenden der DHBW Unterstützung bei ihrem Lernprozess, indem sie Lerninhalte kategorisieren, Lernziele festlegen und ein optimierter Lernplan erstellt wird. Die Funktionen zur Zusammenfassung, Erstellung von Fragen und Tests wurden erfolgreich implementiert. Die App erhielt positives Feedback bei der Präsentation der Studienarbeit und weckte großes Interesse. Der finale Zeitplan zeigt, dass die Implementierung von Zusammenfassungen, Fragen und Tests mehr Zeit in Anspruch nahm als geplant, wodurch die Umsetzung der Sprungfunktion mithilfe von Tags nicht mehr möglich war.	
\end{abstract}

% Schlüsselwörter Deutsch
\begin{keywords}
LearnAhead, Lern-App, Lernplan, Lernziele, Zusammenfassungen, Fragen, Tests, Gamification, Spacing Effect, Tagging, Android
\end{keywords}


\selectlanguage{english}
% Englisches Abstract
\begin{abstract}\noindent 
\end{abstract}
% Schlüsselwörter Englisch
\begin{keywords}

\end{keywords}


\selectlanguage{ngerman}
\newpage