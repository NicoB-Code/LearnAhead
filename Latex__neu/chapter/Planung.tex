\section{Planungsphase}
\label{Planung}

\subsection{Lastenheft}

\subsubsection{Allgemeine Beschreibung}
Eine mobile LernApp, mit der Möglichkeit Lerninhalte zusammenzufassen und von diesen in Abhängigkeit Tests zu erstellen.
Von den eingetragenen Lerninhalten können dann Tests erstellt werden, welche aus verschiedenen Fragen bestehen.
Es besteht die Möglichkeit die Inhalt nach Kategorien zu gruppieren und Lernzielen zu zuweisen.
Mit diesen Lernzielen wird dem User dann ein empfohlener Lernplan erstellt.

\subsubsection{Anforderungen}
\begin{itemize}
    \item Lerninhalte zusammenfassen
    \item Tests erstellen
    \item Lernkategorien
    \item Auf Android im Play Store verfügbar
    \item Lernziele und dazugehörigen automatisch erstellten Lernplan
    \item User-Profil mit Benutzername und Kennwort
\end{itemize}

\subsubsection{Fachliches Umfeld}
\begin{itemize}
    \item Plattformabhängig mit Android Studio
    \item Mobile Lösung
    \item Datenbank
    \item IT-Security
    \item DSGVO
\end{itemize}

\subsubsection{Ausblick}
Bei erfolgreichen Entwicklungsergebnissen soll die Lösung in Betrieb genommen und der Öffentlichkeit, per Play Store, zur Verfügung gestellt werden.

\subsubsection{Erweiterungsmöglichkeiten (optional)}
\begin{itemize}
    \item Importieren und Exportieren von Lerninhalten auf WhatsApp oder Ähnlichem
\end{itemize}


\subsection{Arbeitspaketplan}
\label{sec:arbeitspaketplan}
Der Arbeitspaketplan bezeichnet die Aufzählung jedes Arbeitspakets auf Basis des Lastenhefts. \newline Ein Arbeitspaket wird durch folgendes definiert: 
\begin{itemize}
    \item Ein definiertes Ergebnis (was soll in diesem Arbeitspaket erreicht werden)
    \item Der zeitliche Aufwand des Arbeitspakets
    \item Die Vorbedingungen, die beim Bearbeiten zu beachten sind
    \item Die Dauer
\end{itemize}


\noindent
Um die Arbeitspakete grafisch aufbereitet darstellen zu können, werden diese in die Anwendung \href{https://studienarbeitlernapp.atlassian.net/jira/software/projects/LER/boards/1}{\underline{Jira}}\footnote{\href{https://studienarbeitlernapp.atlassian.net/jira/software/projects/LER/boards/1}{https://studienarbeitlernapp.atlassian.net/jira/software/projects/LER/boards/1}} ausgelagert.
Hier wird der Arbeitspaketplan als Unterteilung der einzelnen Epics in User Stories dargestellt.
Erledigte Epics und User Stories sind dabei unter dem Reiter \gqq{Fertig} einsehbar.
In den einzelnen User Stories wird ein definiertes Ergebnis aus Sicht des Nutzers beschrieben.
Der zeitliche Aufwand der einzelnen Arbeitspakete ergibt sich aus der Summe der Dauer der zugewiesenen Tasks. 
In Kombination mit Scrum werden dabei vor Beginn des jeweiligen Sprints die einzelnen Tasks geschätzt und auf einen festen Arbeitsaufwand festgelegt. 

Im Laufe des Sprints werden dann die zugewiesenen Stunden von den zugewiesenen Bearbeitern abgebaut und im jeweiligen Task aktualisiert.
Ist der Task beendet, so wird er mit \gqq{Done} markiert und besitzt somit keinen Arbeitsaufwand mehr.

Vorbedingungen, sowie die festgelegte Dauer für die Bearbeitung eines Arbeitspaketes werden durch die Sprints definiert. 
Durch Aufteilen der Tasks in verschiedene, nacheinander ablaufende Sprints, können Vorbedingungen durch Einteilung in einen vorangehenden Sprint gesetzt werden.
Darüber hinaus können den einzelnen Tasks Prioritäten zwischen eins und vier zugeordnet werden, was eine Hierarchie innerhalb eines Sprints ermöglicht.
Zusätzlich limitiert die Dauer des Sprints die Bearbeitungszeit für den jeweiligen Task.
\subsection{Zeitplan}
\subsection{Qualitätsmanagement Maßnahmen}
\subsection{Konfigurationsmanagement Maßnahmen}
Die agile Planung im erweitertem Scrum erfolgt in Jira, hier ist der Backlog angelegt, in welchem die Sprint-Planung erfolgt. Meetings werden auf Discord durchgeführt. Die Dokumentation wird mit \LaTeX  geschrieben.\newline
Die Versions- und Releaseverwaltung erfolgt in einem GitHub Repository unter dem Git-Branching-Modell Gitflow. Gitflow sieht zwei Branches vor um den Verlauf des Projekts aufzuzeichnen. Der main-Branch enthält den offiziellen Release-Verlauf und der develop-Branch dient als Integrations-Branch für Features. Der develop-Branch enthält den kompletten Versionsverlauf des Projekts, während der main-Branch eine verkürzte Version enthält.\newline
Jedes neue Feature sollte sich auf seinem eigenen Branch befinden, der zu Sicherungs-/Zusammenarbeitszwecken zum zentralen Repository gepusht werden kann. Ein neuer feature-Branch  wird aus dem develop-branch gemerget. Wenn ein Feature fertig ist, wird es zurück in den develop-Branch gemergt. Features sollten niemals direkt mit dem main-Branch interagieren.\newline
Wenn der develop-Branch genügend Features für ein Release enthält, wird vom develop-Branch ein release-Branch geforkt. Damit beginnt der neue Release-Zyklus. In diesem Branch sollten ab diesem Punkt keine neuen Features mehr hinzugefügt werden, sondern nur Bugfixes und ähnliche Release-orientierte Änderungen. Ist er zur Auslieferung bereit, wird der release-Branch in den main-Branch gemergt und mit einer Versionsnummer getaggt. Zusätzlich wird der release-Branch in den develop-Branch gemerged. \newline
Maintenance- bzw. hotfix-Branches eignen sich für schnelle Patches von Produktions-Releases. Sie werden aus dem main-Branch geforkt. Sobald der Bugfix abgeschlossen ist, wird er sowohl in den main- als auch in den develop-Branch (oder den aktuellen release-Branch) gemergt.
\subsection{Auswahl kritischer Technologien}




