\section{Ergebnisse}
Im folgenden Kapitel werden die Ergebnisse unserer Lern-App präsentiert und aufgelistet.
\subsection{Nutzen der Lern-App}
Der Nutzen unserer geplanten und dahingehend entwickelten Lern-App LearnAhead liegt darin die Studierenden der DHBW bei ihrem Lernprozess zu unterstützen.\\

\noindent
Es können nicht nur Lerninhalte in eine Lernkategorie kategorisiert, sondern auch, mit Hilfe der Lernziele, ihr Zeitlicher Rahmen erfasst werden.\\

\noindent
Die Lerninhalte können mit den Zusammenfassungen einer Lernkategorie mit dem Markdown-Editor erfolgreich erfasst und kondensiert werden.\\
Zu diesen Lerninhalten kann man dann Tests und Fragen erstellen, welche sich auf diese beziehen.

\subsection{Feedback von Präsentation der Studienarbeit}
Im Rahmen der Studienarbeiten wurde uns am 25.05.2023 die Möglichkeit geboten unsere Studienarbeiten in den Räumen N001 bis N004 im Rahmen eines Plakat Forums vorzustellen. \\

\noindent
Hierbei wurde unter anderem die Idee für das weiterreichen der Lerninhalte gegeben, da diese momentan noch von Hand eingegeben werden müssen.\\

\noindent
LearnAhead stoß auf großes Interesse und es wurde vieles hinterfragt, was uns noch umso mehr Ideen einbrachte, von denen einige in den Ausblick wanderten, da die Möglichkeiten bei der Entwicklung einer Lern-App praktisch unendlich sind. Mehr dazu siehe Kapitel \ref{sec:Ausblick}.

\newpage
\subsection{Der finale Zeitplan}
Im Folgenden findet sich der finale Zeitplan, so wie er zum Schluss umgesetzt wurde.\\
\begin{table}[H]
  \centering
  \resizebox{\columnwidth}{!}{%
  \begin{tabular}{l|l|l}
  \multicolumn{1}{c|}{\textbf{Meilenstein}} &
    \multicolumn{1}{c|}{\textbf{Zeitplan}} &
    \multicolumn{1}{c}{\textbf{Beschreibung}} \\ \hline
  Literaturrecherche &
    \begin{tabular}[c]{@{}l@{}}14.10.2022 - \\ 31.01.2023\end{tabular} &
    \begin{tabular}[c]{@{}l@{}}Das Durchführen einer umfangreichen Literaturrecherche\\ auf Basis von wissenschaftlichen Dokumenten.\end{tabular} \\ \hline
  Use-Case-Erstellung &
    \begin{tabular}[c]{@{}l@{}}14.10.2022 - \\ 11.11.2022\end{tabular} &
    \begin{tabular}[c]{@{}l@{}}Identifizierung und Dokumentation der \\ Hauptfunktionalitäten und Anwendungsfälle der Lern-App.\end{tabular} \\ \hline
  UI-Konzept &
    \begin{tabular}[c]{@{}l@{}}11.11.2022 - \\ 02.02.2023\end{tabular} &
    \begin{tabular}[c]{@{}l@{}}Entwicklung eines visuellen Konzepts für die \\ Benutzeroberfläche (UI) der Lern-App.\end{tabular} \\ \hline
  Datenbank-Konzept &
    \begin{tabular}[c]{@{}l@{}}20.01.2023 - \\ 16.02.2023\end{tabular} &
    \begin{tabular}[c]{@{}l@{}}Design und Auswahl des Datenbanksystems, die für die \\ App benötigt wird.\end{tabular} \\ \hline
  Architektur-Konzept &
    \begin{tabular}[c]{@{}l@{}}03.02.2023 -\\ 16.02.2023\end{tabular} &
    \begin{tabular}[c]{@{}l@{}}Realisierung einer Code-Architektur und Auswahl  \\ verschiedener Komponenten sowie Framekworks, \\ die in der App verwendet werden.\end{tabular} \\ \hline
  Architektur-Prototyp &
    \begin{tabular}[c]{@{}l@{}}10.02.2023 - \\ 16.02.2023\end{tabular} &
    \begin{tabular}[c]{@{}l@{}}Erstellung eines ersten Prototypen \\  der die vorgeschlagene Architektur implementiert.\end{tabular} \\ \hline
  Login / Registrierung &
    \begin{tabular}[c]{@{}l@{}}17.02.2023 - \\ 30.03.2023\end{tabular} &
    \begin{tabular}[c]{@{}l@{}}Implementierung der Funktionen für Anmeldung, \\ Registrierung und Passwortwiederherstellung.\end{tabular} \\ \hline
  \begin{tabular}[c]{@{}l@{}}Lernkategorien \& \\ Lernziele\end{tabular} &
    \begin{tabular}[c]{@{}l@{}}31.03.2023 -\\ 11.05.2023\end{tabular} &
    \begin{tabular}[c]{@{}l@{}}Implementierung der Funktion zum Erstellen sowie Verwalten\\  von Lernkategorien und -zielen.\end{tabular} \\ \hline
  \begin{tabular}[c]{@{}l@{}}Erstellung von Fragen \\ und Tests\end{tabular} &
    \begin{tabular}[c]{@{}l@{}}12.05.2023 -\\ 11.07.2023\end{tabular} &
    \begin{tabular}[c]{@{}l@{}}Implementierung der Funktion zum Erstellen sowie Verwalten\\  von Fragen und Tests.\end{tabular} \\ \hline
  \begin{tabular}[c]{@{}l@{}}Erstellung von \\ Zusammenfassungen\end{tabular} &
    \begin{tabular}[c]{@{}l@{}}12.05.2023 -\\ 11.07.2023\end{tabular} &
    \begin{tabular}[c]{@{}l@{}}Implementierung der Funktion zum Erstellen sowie  \\ Verwalten von Zusammenfassungen von Lernkategorien.\end{tabular} \\ \hline
  \begin{tabular}[c]{@{}l@{}}Optimale Pausenberechnung \\ realisieren\end{tabular} &
    \begin{tabular}[c]{@{}l@{}}12.07.2023 -\\ 14.07.2023\end{tabular} &
    \begin{tabular}[c]{@{}l@{}}Erstellung eines Algorithmus, welcher den Nutzer die  \\ optimale Pause vorschlägt sowie errinert.\end{tabular} \\ \hline
  \begin{tabular}[c]{@{}l@{}}Optimale Lernplan \\ generieren\end{tabular} &
    \begin{tabular}[c]{@{}l@{}}12.07.2023 -\\ 14.07.2023\end{tabular} &
    \begin{tabular}[c]{@{}l@{}}Erstellung eines optimalen Lernplans auf Basis der Lernziele.\\\end{tabular} \\ \hline
  Durchführen von User-Tests &
    \begin{tabular}[c]{@{}l@{}}09.07.2023 -\\ 16.07.2023\end{tabular} &
    \begin{tabular}[c]{@{}l@{}}Durchführung von umfassenden Tests, um die Qualität, \\ Funktionalität und Stabilität der App sicherzustellen.\end{tabular} \\ \hline
  Bugs beheben &
    \begin{tabular}[c]{@{}l@{}}14.07.2023 - \\ 16.07.2023\end{tabular} &
    \begin{tabular}[c]{@{}l@{}}Behebung von Fehlern und Problemen in der App.\end{tabular} \\ \hline
  Dokumentation &
    \begin{tabular}[c]{@{}l@{}}14.10.2022 - \\ 16.07.2023\end{tabular} &
    \begin{tabular}[c]{@{}l@{}}Erstellung einer wissenschaftlichen Arbeit, die das Vorgehen, \\ Funktionen, die Implementierung sowie die Verwendung \\ der App begründet.\end{tabular}
  \end{tabular}%
  }
  \end{table}

  \noindent
  Anhand des finalen Zeitplans lässt sich erkennen, dass die Implementierung von den Zusammenfassungen, Fragen und Tests mehr Zeit in Anspruch genommen hat, als ursprünglich geplant. Daraus folgt, dass das \gqq{Springen} mithilfe von Tags nicht mehr implementiert werden konnte.