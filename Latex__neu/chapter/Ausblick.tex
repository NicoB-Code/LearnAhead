\section{Fazit}
LernAhead ist ein Projekt, welches von uns erfolgreich abgeschlossen wurde, jedoch noch einen Weg vor sich hat, der sich erst herauskristallisieren muss.\\

\noindent
Im folgenden Abschnitt wird der Ablauf des Projekts kritisch betrachtet sowie mögliche zukünftige Features erwägt und daraus ein Resümee gezogen.

\subsection{Ablauf des Projekts}
Im Rahmen der Studienarbeit sollen die Studierenden sich \gqq{in ein komplexes, aber eng umgrenztes Gebiet vertiefend einarbeiten und den allgemeinen Stand des Wissens erwerben.}\cite{DHBW_Studienarbeit}\\

\noindent
Am Beginn von LearnAhead wurden uns von der zuständigen Betreuerin Frau Claudia Zinser zwei wissenschaftliche Bücher ausgehändigt, aus welchen wir Informationen über neuronalen Lernens und andere dem Lernen zugeschriebene Gebiete erfassten.\\

\noindent
Diese Masse an Wissen musste erst einmal neben dem eigentlichen Hochschulbetrieb verarbeitet werden. Dies zog sich bis in die Mitte der 5. Praxisphase, da nicht nur die beiden ausgehändigten Bücher gelesen wurden, sondern auch andere Dokumente Teil unserer Literaturrecherche waren.\\

\noindent
Bei der kompletten Durchführung der Studienarbeit war es so, dass diese immer einen kleinen Zeitslot benötigte, egal wann, da es sich hierbei ja um ein kontinuierliches Projekt handelte.\\

\noindent
Es war interessant und äußerst lehrreich ein Projekt zu erleben, welches von Grundauf, von Anfang bis Ende geplant und implementiert wurde.\\

\noindent
Wir sind mit dem Ablauf der Studienarbeit sehr zufrieden und haben es eingesehen, dass wir an manchen Stellen bei der Planung überambitioniert waren, jedoch freuen wir uns, das Projekt an kommende Generationen von Studenten weiterzugeben und das Projekt wachsen zu sehen.\\

\noindent
LearnAhead war kein Projekt, was mit bereits vorhandenem Wissen aus dem Boden gestampft wurde, es wurden uns an vielen Stellen über den Prozess des Lernens, nach nun mehr als fast 15 Jahren Lernen in allen möglichen Schulen, die Augen geöffnet.\\ 

\noindent
Abseits der Hochschule haben wir nun ein umfassendes Interesse an dem neurologischen Prozess des Lernens, da sich hierfür auch sehr viele Erkenntnisse über maschinelles Lernen ziehen lassen.

\subsection{Ausblick und die Zukunft}\label{sec:Ausblick}
Mit der Veröffentlichung von LearnAhead geht ein kleines Kapitel in unserer Geschichte zu Ende, aber ein anderes Tor wird noch viel weiter aufgestoßen, welches in die Zukunft von LearnAhead führt.\\
Viele Features könnten noch in LearnAhead implementiert werden, im nachfolgenden Abschnitt werden einige davon aufgelistet.\\
\noindent
\begin{enumerate}
    \item Lerninhalte, sowie dessen Tests und Fragen müssen momentan noch manuell eingegeben werden. Dies hat zwar den Vorteil, dass hierdurch die Verbindungen im Gehirn des Lernenden gefestigt werden.\\
    
    \noindent
    Für die Zukunft gäbe es noch die Möglichkeit die App in so einer Form zu erweitern, dass Lerninhalte und Tests oder ganze Lernkategorien importiert, beziehungsweise exportiert werden können, so dass ein neuer Jahrgang den vergangen Jahrgang nach der Lernkategorie für die anstehende Klausur fragen könnte.
    \item Ein anderes Team aus Studierenden haben im Rahmen ihrer Studienarbeit eine Lern-App entwickelt, welche ihren Fokus auf kollaboratives Lernen legt. In Absprache mit der zuständigen Betreuerin Claudia Zinser könnten hierbei beide Lern-Apps über eine API oder Ähnliches miteinander verbunden werden.
    \item Bei den Zusammenfassungen ist es aktuell noch mühsam die Inhalte selber von Grund auf einzugeben. Um dem entgegenzuwirken könnte ein Template-System aufgesetzt werden, welches es den Nutzern ermöglicht Vorlagen für Zusammenfassungen zu erstellen.
    \item Bisher wird dem User das Lernen nur durch einige zufällig ausgewählte Lerntipps beigebracht. Unsere Vision am Anfang von LearnAhead war ein Bereich der App, in welchem man das Lernen lernen kann und allgemeine Falschannahmen zum lernen aufgeklärt werden.
    \item Der User wird dafür belohnt, wenn er eine Frage nach mehrmaligem falsch beantworten richtig beantwortet. Es könnte zur Unterstützung des Lernprozesses nun ein Algorithmus implementiert werden, welcher den User nach mehrmaligem falsch beantworten einer Frage auf die zugehörige Zusammenfassung hinleitet.\\

    \noindent
    Hierfür müssten Zusammenfassungsinhalte mit Fragen verknüpft werden. Dies könnte beispielsweise über die Tag-Funktion geschehen, indem man einen Tag die Zugehörigkeit zu einer Zusammenfassung zuspricht.
    \item Der Markdown-Editor unterstützt bisher nur Basic-Markdown Syntax. Markwon stellt die Möglichkeit zur Verfügung Plugins einzubauen, welche diese Syntax erweitern. Es könnte eine Funktion zum einfügen von mathematischen Operationen implementiert werden, wie es MathJax (\href{https://www.mathjax.org/}{https://www.mathjax.org/}) in LaTeX implementiert.
    \item Momentan wird der Lernfortschritt einer Lernkategorie dem Benutzer nicht vollständig ersichtlich. In der weiteren Entwicklung des Projektes könnte deswegen ein Lernfortschritts-Balken implementiert werden, welcher dem User seinen momentan Lernfortschritt einer Lernkategorie anzeigt.
    \item Tags sind momentan statisch anlegbar. Dies bedeutet, dass wenn ein Tag erstellt wird und dabei ein Fehler unterläuft, muss dieser gelöscht und korrekt neu erstellt werden. Um dem entgegenzuwirken könnte man ein Tag-Bearbeitungssystem entwickeln.
    \item LearnAhead ist momentan durch seine Plattform-Abhängigkeit nur auf Android Mobilgeräte beschränkt. Um dies zu verbessern könnte LearnAhead zu einer \ac{PWA} umfunktioniert werden, um Plattformunabhängigkeit zu ermöglichen.
    \item Derzeit muss der User sich die Daten für seine Lernziele quantitativ merken und selbst verwalten. Dieses Problem könnte mit einem Kalender-System gelöst werden, welches dem User eine bessere zeitliche Übersicht über seine Lernziele gibt.
    \item Wir haben uns entschieden, unsere App nicht im Google Play Store zu veröffentlichen, da dies im Rahmen unseres Studiums nicht erforderlich ist. Eine der Herausforderungen bei der Veröffentlichung einer App im Play Store ist die Notwendigkeit einer umfassenden Datenschutzerklärung, sowie das dreimalige nicht akzeptieren von gültigen amtlichen Ausweisdokumenten seitens Google. Diese zu erstellen und sicherzustellen, dass unsere App vollständig konform ist, würde zusätzliche Zeit und Ressourcen erfordern. Da unser Hauptziel darin besteht, die technischen sowie wissenschaftlichen Fähigkeiten und Kenntnisse zu erwerben, die für die Entwicklung einer funktionsfähigen App erforderlich sind, haben wir uns darauf konzentriert, eine App zu erstellen, die unseren Anforderungen entspricht, anstatt sie für eine breitere Öffentlichkeit zugänglich zu machen. Um die App für die breite Öffentlichkeit zugänglich zu machen, könnte man diesen Schritt noch in der Zukunft von LearnAhead gehen.
\end{enumerate}
