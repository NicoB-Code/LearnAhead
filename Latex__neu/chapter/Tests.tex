\section{User-Tests}
Um Fehlverhalten und mögliche Verständnisprobleme beim Benutzen der App aufzudecken, wurden User Tests durchgeführt. Hierbei wird explizit darauf geachtet, dass nicht nur frei getestet wird, sondern die Tester anhand definierter Aufgaben und Ziele alle Funktionalitäten durchlaufen. \newline
Die User Tests werden deshalb in:
\begin{enumerate}
    \item Offene Tests, allgemeine Ziele
    \item Freie Tests
\end{enumerate}
unterteilt.\\ Im Folgenden findet sich der ausführliche Testbogen, welcher an die Test-User verschickt wurde.\\\\
\textit{
    Danke, dass du dich entschieden hast uns zu helfen und mit uns unsere App zu testen.
    Höchstwahrscheinlich wurdest du von einem von uns beiden angesprochen und gebeten zu helfen.
    Deswegen kann es sein, dass du nicht beide von uns kennst, wir sind: Phillipp Patzelt und Nico Bayer.
    Alle samt Studenten im 6. Semester an der DHBW Ravensburg.\\\\
    Zusammen haben wir im letzten dreiviertel Jahr eine Lern-App erdacht, geplant und erstellt im Rahmen wissenschaftlichen Arbeit.
    Nun sind wir im Endspurt und stehen kurz vor dem offiziellen Release von LearnAhead, einer mobilen Lern-Applikation.\\\\
    Doch nun kurz vor der Veröffentlichung unserer App müssen wir ein letztes mal testen ob auch alles klappt. Hier kommst du ins spiel.
    Bei diesem letzten Test benötigen wir Hilfe. Es kann sein, dass uns durch das montaelange Arbeiten an unserer App, grobe Fehler gar nicht mehr auffallen oder Teile unserer App unpraktisch oder nicht intuitiv zu bedienen sind.\\\\
    Deswegen wurdest du ausgewählt vor dem regulären Betrieb unsere App zu testen und währenddessen diesen Bericht auszufüllen. Bitte gib uns diesen Bogen ausgefüllt bis zum 30.06.2023 zurück, damit wir püntklich vor der Abgabe und dem Release von LearnAhead am 17.07.2023 die letzten Fehler korrigieren können.\\\\
    Bitte scheue dich nicht uns ein ehrliches Feedback zu geben und wir wünschen im Namen von uns beiden viel Spaß und Erfolg bei diesen Tests!
}
\subsection{Offene Tests, allgemeine Ziele}
Das Ziel der offenen Tests ist es, durch eine bewusst grob gewählte Aufgabenstellung, die App auf Praktibilität zu testen, zu sehen ob der Nutzer die App etwas stören könnte und wie intuitiv er sich in der App bewegt.\newline
\textbf{Aufgabe:}\newline
Bitte rufe unsere App auf deinem Mobilgerät auf, falls du sie noch nicht installiert hast, tue dies, indem du die \gqq{app-debug.apk} vom bereitgestellten Google-Drive Link herunterziehst und installierst.\\
Du wirst nun auf der Login-Seite landen. Deine erste Aufgabe ist nun auf eigene Faust Folgendes zu tun: \textbf{Eine Zusammenfassung zu erstellen}.\\Versuche von selbst die dafür notwendigen Schritte zu erschließen und auszuführen.\\
Bitte notiere dafür im Folgenden dein Mobilgerät, sowie alles was dir auffällt. Konntest du die Aufgabe leicht ausführen? Gabs es irgendwelche Fehler oder kamst du an irgendeinem Punkt an, an dem du nicht mehr weiterwusstest?\\
Notiere dazu bitte nur konrekt die Antwort auf diese Fragen. Gib bitte erst im nächsten Abschnitt dann dein Feedback oder die Dinge, welche dir in deinen Augen als unpraktisch auffallen.\\
Wenn du deine Aufgabe geschafft hast, verbleibe bitte im Edit-Fenster der Zusammenfassung.\\
Genutztes Mobilgerät: \rule{0.25\textwidth}{0.4pt}\\\\
\textbf{Antwort auf die obige Frage:}\\
\noindent\rule{\textwidth}{0.4pt}\\
\noindent\rule{\textwidth}{0.4pt}\\
\noindent\rule{\textwidth}{0.4pt}\\
\noindent\rule{\textwidth}{0.4pt}\\
\noindent\rule{\textwidth}{0.4pt}\\
\noindent\rule{\textwidth}{0.4pt}\\
\noindent\rule{\textwidth}{0.4pt}\\
\noindent\rule{\textwidth}{0.4pt}\\
\textbf{Feedback zum ersten Eindruck, Wünsche und Dinge die dir unpraktisch vorkommen:}
\noindent\rule{\textwidth}{0.4pt}\\
\noindent\rule{\textwidth}{0.4pt}\\
\noindent\rule{\textwidth}{0.4pt}\\
\noindent\rule{\textwidth}{0.4pt}\\
\noindent\rule{\textwidth}{0.4pt}\\
\noindent\rule{\textwidth}{0.4pt}\\\\
Sehr schön, nun befindest du dich in einem Kernteil unserer App, den Zusammenfassungen. Deine nächste Aufgabe ist es: \textbf{Einen kleinen Markdown Text verfassen und ihn dir anzeigen lassen}.\\
Falls du mit Markdown noch nicht vertraut bist, kein Problem! Im Internet findest du viele Anleitungen wie Markdown funktioniert, hier ein Beispiel: https://www.markdownguide.org/ \\
Folge hierbei der gleichen Prozedur wie vorhin, suche dir selbst einen Weg durch die App. Fallen dir Fehler auf? Wusstest du wo du suchen sollst um die Aufgabe zu erledigen? Im Folgenden darfst du alles was dir auffällt UND dein Feedback und Kritik auf einmal aufschreiben.\\
\noindent\rule{\textwidth}{0.4pt}\\
\noindent\rule{\textwidth}{0.4pt}\\
\noindent\rule{\textwidth}{0.4pt}\\
\noindent\rule{\textwidth}{0.4pt}\\
\noindent\rule{\textwidth}{0.4pt}\\
\noindent\rule{\textwidth}{0.4pt}\\
\noindent\rule{\textwidth}{0.4pt}\\
\noindent\rule{\textwidth}{0.4pt}\\\\
Gut! Nun widmest du dich einem neuen Teil der App und somit auch einer neuen Aufgabe: \textbf{Ändere dein Profilbild}. \\
Wie zuvor, bitte notiere Fehler und Dinge die dir als unpraktisch auffallen.\\
\noindent\rule{\textwidth}{0.4pt}\\
\noindent\rule{\textwidth}{0.4pt}\\
\noindent\rule{\textwidth}{0.4pt}\\
\noindent\rule{\textwidth}{0.4pt}\\
\noindent\rule{\textwidth}{0.4pt}\\
\noindent\rule{\textwidth}{0.4pt}\\
\noindent\rule{\textwidth}{0.4pt}\\\\
Wir sind fast beim letzten Teil der offenen Tests angelangt, nun wird deine Aufgabe sein: \textbf{Ein Lernziel zu erstellen und zu editieren}.\\
Notiere auch hier bitte alle Fehler und alles unpraktische was dir auffällt.\\
\noindent\rule{\textwidth}{0.4pt}\\
\noindent\rule{\textwidth}{0.4pt}\\
\noindent\rule{\textwidth}{0.4pt}\\
\noindent\rule{\textwidth}{0.4pt}\\
\noindent\rule{\textwidth}{0.4pt}\\
\noindent\rule{\textwidth}{0.4pt}\\\\
Beim letzten Teil der offenen Tests geht es darum, \textbf{deine erstellten Elemente zu löschen und dich auszuloggen}.\\
Wie bereits im vorherigen Ablauf notiere alle Fehler und unpraktischen Elemente.\\
\noindent\rule{\textwidth}{0.4pt}\\
\noindent\rule{\textwidth}{0.4pt}\\
\noindent\rule{\textwidth}{0.4pt}\\
\noindent\rule{\textwidth}{0.4pt}\\
\noindent\rule{\textwidth}{0.4pt}\\
\noindent\rule{\textwidth}{0.4pt}\\\\
\subsection{Freie Tests}
In den freien Tests geht es darum, dass du selbst gezielt versuchst etwas kaputt zu machen oder Fehler zu produzieren. Teste bitte auf vielen unterschiedlichen Geräten, ob alles funktioniert und du noch irgendetwas entdeckst, das du bisher noch nicht erwähnt hast. \\
Falls du tatsächlich noch etwas findest, beschreibe uns bitte genau, wie es dazu kam, auf welcher Seite du warst, welches Gerät du genutzt hast und was du geklickt hast.\\
Des Weiteren erwähne hier auch alle noch nicht genannten Verbesserungsvorschläge an unsere App und wir werden sehen, was wir tun können, um diese so schnell wie möglich zu implementieren. Danke und viel Erfolg!\\
\noindent\rule{\textwidth}{0.4pt}\\ % This creates a horizontal line
\noindent\rule{\textwidth}{0.4pt}\\
\noindent\rule{\textwidth}{0.4pt}\\
\noindent\rule{\textwidth}{0.4pt}\\
