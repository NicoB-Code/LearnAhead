\section{Einleitung}
\gqq{Es ist keine Schande nichts zu wissen, wohl aber nichts lernen zu wollen.}\\
Diese Worte verwendete schon der griechische Philosoph Platon. Wir hören nie auf zu lernen, das ist ein Fakt. \\
Der Prozess des Lernens lässt sich nicht aktiv unterbinden. Lernen ist der Impuls, Informationen zu verarbeiten, einzuordnen und zu verstehen. Das Behalten des Erlernten passiert ebenfalls automatisch, wenn auch bei vielen Dingen nicht dauerhaft. Der Mensch ist so angelegt, dass er seine Umwelt begreifen will. Somit ist Lernen ein intrinsisch motivierter Prozess, der zwar angeregt, aber nicht verordnet werden kann.
Etwa 100 Milliarden Nervenzellen besitzt unser Gehirn, die sogenannten Neuronen. An deren Ende liegen Synapsen, die elektrische Signale der Neuronen in Form von chemischen Botenstoffen an die nächsten Neuronen abgegeben. Diese Kettenreaktionen tragen die Informationen und Signale durch unser neuronales Netz im Gehirn an die Stellen, wo es benötigt wird, egal ob wir unsere Muskeln bewegen wollen oder unsere Sinne einsetzen, und lösen dort Reaktionen aus. \\
Obwohl der Mensch also nicht nur in der Theorie, sondern auch in der Praxis für das Lernen konstruiert ist, macht das Lernen vielen Menschen nicht spaß und sie behaupten: \gqq{Sie können nicht lernen}. \\
Faktisch gesehen ist das jedoch nicht richtig! Jahrhunderte, ja sogar Jahrtausende der Evolution haben uns dazu ausgebildet nie aufzuhören zu lernen. Die heutige Welt bietet jedoch viele Tücken, welche das Lernen einschränken können. \\
Das Lernen ist ein komplexer, jedoch kein komplizierter Prozess, es muss nicht der komplette Bio-chemische Ablauf verstanden werden, jedoch sollte man einige wichtige Grundlagen beachten.