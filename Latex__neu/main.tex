%%%%%%%%%%%%%%%%%%%%%%%%%%%%%%%%%%%%%%%%%%%%%%%%%%%%%%%%%%%%%%%%%%%%
%%%           Vorlage für eine Ausarbeitung an der DHBW          %%%
%%%                                                              %%%
%%%      Bereiche die bearbeitet werden müssen werden durch      %%%
%%%      einen solchen Kommentarblock eingeleitet und enden      %%%
%%%      mit der nächsten Trennlinie.                            %%%
%%%                                                              %%%
%%%      In dieser Datei müssen folgende Bereiche bearbeitet     %%%
%%%      werden:                                                 %%%
%%%      - Angaben zur Arbeit                                    %%%
%%%      - EIGENE KAPITEL EINFÜGEN                               %%%
%%%                                                              %%%
%%%      Benötigte Seiten und Verzeichnisse können unter         %%%
%%%      "Einführung und Verzeichnisse" ein- bzw. auskommentiert %%%
%%%      werden.                                                 %%%
%%%                                                              %%%
%%%%%%%%%%%%%%%%%%%%%%%%%%%%%%%%%%%%%%%%%%%%%%%%%%%%%%%%%%%%%%%%%%%%

\documentclass[a4paper,12pt]{article}
\usepackage[left=2.5cm,right=2.5cm,top=2.5cm,bottom=2.5cm,includehead]{geometry}      % Einstellungen der Seitenränder
\usepackage[english, ngerman]{babel}                                                  % deutsche Silbentrennung
\usepackage[utf8]{inputenc}                                                           % Umlaute
\usepackage[T1]{fontenc}													                                    % Umlaute auch richtig ausgeben
\usepackage{newtxtext,newtxmath}                                                      % Font = Times New Roman
\usepackage{hyperref}
%\usepackage[nottoc]{tocbibind}
\usepackage{fancyhdr}
\usepackage{setspace}
\usepackage[backend=bibtex, style=numeric,sortcites,sorting=nty,backref,natbib,hyperref]{biblatex}      % Bibliothek für Zitate
\usepackage{csquotes}                                                                 % Zusatzpacket für Zitate
\usepackage{amsmath}                                                                  % Zurücksetzen der Tabellen- und Abbildungsnummerierung je Sektion
\usepackage[labelfont=bf]{caption}                                      % Bild- und Tabellenunterschrift (fett)
\usepackage[bottom,multiple,hang,marginal]{footmisc}                                  % Fußnoten [Ausrichtung unten, Trennung durch Seperator bei mehreren Fußnoten]
\usepackage{graphicx}  
\graphicspath{{./images/}}                                                            % Grafiken
\usepackage[dvipsnames]{xcolor}                                                       % Farbige Buchstaben
\usepackage{wrapfig}                                                                  % Bilder in Text integrieren
\usepackage{enumitem}                                                                 % Befehl setlist (Zeilenabstand für itemize Umgebung auf 1 setzen)
\usepackage{listings, chngcntr}                                                                 % Quelltexte
\usepackage{subcaption}
\definecolor{lightgray}{rgb}{.9,.9,.9}
\definecolor{darkgray}{rgb}{.4,.4,.4}
\definecolor{purple}{rgb}{0.65, 0.12, 0.82}
\lstdefinelanguage{JavaScript}{
	keywords={typeof, new, true, false, catch, function, return, null, catch, switch, var, if, in, while, do, else, case, break,let},
	ndkeywords={class, export, boolean, throw, implements, import, this},
	comment=[l]{//},
	morecomment=[s]{/*}{*/},
	morestring=[b]',
	morestring=[b]"
}
\lstset{
	keywordstyle=\color{blue}\bfseries,
	ndkeywordstyle=\color{darkgray}\bfseries,
	identifierstyle=\color{black},
	sensitive=false,
	commentstyle=\color{purple}\ttfamily,
	stringstyle=\color{red}\ttfamily,
	backgroundcolor=\color{lightgray},
	extendedchars=true,
	basicstyle=\footnotesize\ttfamily,
	showstringspaces=false,
	showspaces=false,
	numbers=left,
	numberstyle=\footnotesize,
	numbersep=9pt,
	tabsize=2,
	breaklines=true,
	showtabs=false,
	captionpos=b
}
\usepackage{tabularx}                                                                 % Tabellen
\usepackage[nohyperlinks, printonlyused, withpage]{acronym}                           % Abkürzungen
\usepackage{dirtree}      
\usepackage{xcolor}
\usepackage{float}
\usepackage{rotating}
\usepackage{tocbibind}	% Inhaltsverzeichnis Eintrag im Inhaltsverzeichnis
\usepackage{tikz}
\usetikzlibrary{positioning, shapes, calc}
%%%%%%%%%%%%%%%%%%%%%%%%%%%%%%%%%%%%%%%%%%%%%%%%%%%%%%%%%%%%%%%%%%%%
%%%                      Angaben zur Arbeit                      %%%
%%%%%%%%%%%%%%%%%%%%%%%%%%%%%%%%%%%%%%%%%%%%%%%%%%%%%%%%%%%%%%%%%%%%
\def\vFirmenlogoPfad{images/Hensoldt\_Logo.jpg}                        %% relativer Pfad Bsp.: images/Firmenlogo.png
\def\vDHBWLogoPfad{images/DHBW\_logo.jpg}                          %% relativer Pfad Bsp.: images/DHBW_logo.jpg

\def\vTitel{DataInspector - Erweiterung einer abteilungsspezifischen C++ Library}                           %% 
\def\vUntertitel{}                      %% 
\def\vArbeitstyp{Bachelorarbeit}                      %% Projektarbeit/Seminararbeit/Bachelorarbeit

\def\vAutor{Nico Bayer}                           %% Vorname Nachname
\def\vMatrikelnummer{3056312}                  %% 7-stellige Zahl
\def\vKursKuerzel{TIT20}                     %% Bsp.: TIT20
\def\vPhasenbezeichnung{Praxisphase}               %% Praxisphase/Theoriephase
\def\vStudienJahr{dritte}                     %% erste/zweite/dritte
\def\vDHBWStandort{Ravensburg}                    %% Bsp.: Ravensburg
\def\vDHBWCampus{Friedrichshafen}                      %% Bsp.: Friedrichshafen
\def\vFakultaet{Technik}                       %% Technik/Wirtschaft
\def\vStudiengang{Informationstechnik}                     %% Informationstechnik/...

\def\vBetrieb{Hensoldt Sensors GmbH}                         %% 
\def\vBearbeitungsort{89077 Ulm}                 %% 
\def\vAbteilung{HEYD4}                       %% 
\def\vBetreuer{Dipl. Phys. Ulrich Hierl}                        %% Vorname Nachname

\def\vAbgabedatum{Frau Schmidt Mail nachfragen}               %% DD. MONTH YYYY
\def\vBearbeitungszeitraum{11 Wochen, KW 2-13}            %% DD.MM.YYYY - DD.MM.YYYY


%%%%%%%%%%%%%%%%%%%%%%%%% Eigene Kommandos %%%%%%%%%%%%%%%%%%%%%%%%%
% Definition von \gqq{}: Text in Anführungszeichen
\newcommand{\gqq}[1]{\glqq #1\grqq}
% Spezielle Hervorhebung von Schlüsselwörtern
\newcommand{\textOrdner}[1]{\texttt{#1}}
\newcommand{\textVariable}[1]{\texttt{#1}}
\newcommand{\textKlasse}[1]{\texttt{#1}}
\newcommand{\textFunktion}[1]{\texttt{#1}}


%%%%%%%%%%%%%%%%%%%% Zitatbibliothek einbinden %%%%%%%%%%%%%%%%%%%%%
\addbibresource{./literatur/literatur.bib}


%%%%%%%%%%%%%%%%%%%%%%%% PDF-Einstellungen %%%%%%%%%%%%%%%%%%%%%%%%%
\hypersetup{
	bookmarksopen=false,
	bookmarksnumbered=true,
	bookmarksopenlevel=0,
	pdftitle=\vTitel,
	pdfsubject=\vTitel,
	pdfauthor=\vAutor,
	pdfborder={0 0 0},
	pdfstartview=Fit,
	pdfpagelayout=SinglePage
}

%%%%%%%%%%%%%%%%%%%%%%%% Kopf- und Fußzeile %%%%%%%%%%%%%%%%%%%%%%%%
\pagestyle{fancy}
\setlength{\headheight}{15pt}
\fancyhf{}
\fancyhead[R]{\thepage}


%%%%%%%%%%%%%%%%%%%%%%%%%%%%%% Layout %%%%%%%%%%%%%%%%%%%%%%%%%%%%%%
\onehalfspacing
\setlist{noitemsep}
\widowpenalties=3 10000 10000 150	% Umbrüche

\addto\captionsngerman{
  \renewcommand{\figurename}{Abb.}
  \renewcommand{\tablename}{Tab.}
}
\renewcommand{\thetable}{\arabic{section}.\arabic{table}}   % Tabellennummerierung mit Section
\renewcommand{\thefigure}{\arabic{section}.\arabic{figure}} % Abbildungsnummerierung mit Section
\renewcommand{\thefootnote}{\arabic{footnote}}              % Sektionsbezeichnung von Fußnoten entfernen

\renewcommand{\multfootsep}{, }                             % Mehrere Fußnoten durch ", " trennen


%%%%%%%%%%%%%%%%%%%%%%%%%%%%% Dokument %%%%%%%%%%%%%%%%%%%%%%%%%%%%%

\begin{document}
\counterwithin{lstlisting}{subsection} 						% Muss nach \begin{document} stehen, da ansonsten nicht bekannt
\counterwithin{table}{subsection}                               % Tabellennummerierung je Sektion zurücksetzen
\counterwithin{figure}{subsection}                              % Abbildungsnummerierung je Sektion zurücksetzen
\newsavebox{\savefig}

  %%%%%%%%%%%%%%%%%%% Einführung und Verzeichnisse %%%%%%%%%%%%%%%%%%%
  \pagenumbering{Roman}

  \begin{titlepage}
  \begin{minipage}{6in}
    \vspace*{-2cm}
    \centering
    \hspace{-2cm}
	\ifx\vFirmenlogoPfad\empty
	\else
    \raisebox{-0.5\height}{\includegraphics[height=3cm]{\vFirmenlogoPfad}}
  \fi
	\hfill
	\ifx\vDHBWLogoPfad\empty
	\else
   	\raisebox{-0.5\height}{\includegraphics[height=3cm]{\vDHBWLogoPfad}}
	\fi
  \end{minipage}
  \begin{center}
    \vspace*{0.5cm}
    \Huge\textbf{\vTitel}\\
		\ifx\vUntertitel\empty
		\else
			\Large\rm\vUntertitel\\
		\fi
		\vspace*{2cm}
		\textbf{\vArbeitstyp}\\
		\normalsize
		\vspace*{1.3cm}
		an der Fakultät für \vFakultaet\\
		im Studiengang \vStudiengang\\
		\vspace*{1cm}
		an der \ac{DHBW} \vDHBWStandort\\
		\ifx\vDHBWCampus\empty
		\else
		Campus \vDHBWCampus\\
		\fi
		\vspace*{1cm}
		von\\
		\ifx\vAutor\empty
		\else
			\vAutor\\
		\fi
		\vspace*{2cm}
		\vfill
  
	  \begin{tabular}{ll}
	    Bearbeitungszeitraum:&\vBearbeitungszeitraum\\
	    Matrikelnummer, Kurs:&\vMatrikelnummer, \vKursKuerzel\\
		  Betreuerin der Studienarbeit:&\vBetreuer\\
	  \end{tabular}
	\end{center}
\end{titlepage}
\newpage
\setcounter{page}{2}
  % \thispagestyle{empty}
\section*{\Huge{Sperrvermerk}}

\addcontentsline{toc}{section}{Sperrvermerk}
gemäß Ziffer 1.1.13 der Anlage 1 zu §§ 3, 4 und 5  der Studien- und Prüfungsordnung für die Bachelorstudiengänge im Studienbereich Technik der Dualen Hochschule Baden-Würt­tem­berg vom 29.09.2017.\\

\noindent \gqq{Der Inhalt dieser Arbeit darf weder als Ganzes noch in Auszügen Personen außerhalb des Prüfungsprozesses und des Evaluationsverfahrens zugänglich gemacht werden, sofern keine anders lautende Genehmigung vom Dualen Partner vorliegt.}

\vfill
\leavevmode
\newline
\parbox{6cm}{\strut\centering \vBearbeitungsort, \vAbgabedatum\hrule\strut\centering\footnotesize Ort, Datum} 
\hfill
\ifx\vUnterschrift\empty
\parbox{6cm}{\strut\hspace{1pt} \vAbteilung\hrule\strut\centering\footnotesize Abteilung, Unterschrift}
\else
\parbox{6cm}{\strut\hspace{1pt} \vAbteilung, \parbox[b]{3cm}{\vspace{-10cm}\includegraphics[width=3cm]{\vUnterschrift}}\hrule\strut\centering\footnotesize Abteilung, Unterschrift}
\fi
\vspace{1cm}

\newpage
  \thispagestyle{empty}
\section*{\Huge{Selbstständigkeitserklärung}}

\addcontentsline{toc}{section}{Selbstständigkeitserklärung}
gemäß Ziffer 1.1.13 der Anlage 1 zu §§ 3, 4 und 5  der Studien- und Prüfungsordnung für die Bachelorstudiengänge im Studienbereich Technik der Dualen Hochschule Baden-Würt­tem­berg vom 29.09.2017.

\noindent Wir, Phillipp Patzelt und Nico Bayer versicheren hiermit, dass wir unsere Studienarbeit zu dem Thema: Entwicklung einer mobilen Lern-Applikation
\begin{center}
	\Large\textbf{\vTitel}
\end{center}
selbstständig verfasst und keine anderen als die angegebenen Quellen und Hilfsmittel benutzt haben. Wir versicheren zudem, dass die eingereichte elektronische Fassung mit der gedruckten Fassung übereinstimmt.

\vfill
\leavevmode
\newline
\parbox{6cm}{\strut\centering \vBearbeitungsort, \vAbgabedatum\hrule\strut\centering\footnotesize Ort, Datum} 
\hfill
\ifx\vUnterschrift\empty
\parbox{6cm}{\strut\hspace{1pt} \vAbteilung\hrule\strut\centering\footnotesize Abteilung, Unterschrift}
\else
\parbox{6cm}{\strut\hspace{1pt} \vAbteilung, \parbox[b]{3cm}{\vspace{-10cm}\includegraphics[width=3cm]{\vUnterschrift}}\hrule\strut\centering\footnotesize Abteilung, Unterschrift}
\fi
\vspace{1cm}

\newpage
  \phantomsection
\newenvironment{keywords}{
	\begin{flushleft}
	\small	
	\textbf{
		\iflanguage{ngerman}{Schlüsselwörter}{\iflanguage{english}{Keywords}{}}
	}
}{\end{flushleft}}

% Deutsche Zusammenfassung
\begin{abstract}
	
\end{abstract}

% Schlüsselwörter Deutsch
\begin{keywords}
	
\end{keywords}


\selectlanguage{english}
% Englisches Abstract
\begin{abstract}

\end{abstract}

% Schlüsselwörter Englisch
\begin{keywords}

\end{keywords}


\selectlanguage{ngerman}
\newpage
  \tableofcontents
\newpage
  \section*{Abkürzungsverzeichnis}
\addcontentsline{toc}{section}{Abkürzungsverzeichnis}
\begin{acronym}
  \acro{DHBW}{Duale Hochschule Ba\-den-\-Würt\-tem\-berg}  
\end{acronym}
\newpage
  \listoffigures
\listoftables
\newpage
  %\listoftables
\newpage
  \lstlistoflistings
\addcontentsline{toc}{section}{Listings}
\newpage
  % \section*{Vorwort}
\addcontentsline{toc}{section}{Vorwort}
\newpage


  %%%%%%%%%%%%%%%%%%%%%%%%%%%%% Kapitel %%%%%%%%%%%%%%%%%%%%%%%%%%%%%%
  \pagestyle{fancy}
  \fancyhead[L]{\nouppercase{\rightmark}}    % Abschnittsname im Header
  \pagenumbering{arabic}

  %%%%%%%%%%%%%%%%%%%%%%%%%%%%%%%%%%%%%%%%%%%%%%%%%%%%%%%%%%%%%%%%%%%%
  %%%%                   EIGENE KAPITEL EINFÜGEN                  %%%%
  %%%%%%%%%%%%%%%%%%%%%%%%%%%%%%%%%%%%%%%%%%%%%%%%%%%%%%%%%%%%%%%%%%%%
  \section{Einleitung}
test Einleitung
  \include{chapter/Aufgabenstellung}
  \include{chapter/Stand_der_Technik}
  \include{chapter/Loesungsansatz}
  \include{chapter/Testen}
  \section{Fazit}
LernAhead ist ein Projekt, welches von uns erfolgreich abgeschlossen wurde, jedoch noch einen Weg vor sich hat, der sich erst herauskristallisieren muss.\\
Im folgenden Abschnitt wird der Ablauf des Projekts kritisch betrachtet sowie mögliche zukünftige Features erwägt und daraus ein Resümee gezogen.\\
\subsection{Ablauf des Projekts}
Im Rahmen der Studienarbeit sollen die Studierenden sich \gqq{in ein komplexes, aber eng umgrenztes Gebiet vertiefend einarbeiten und den allgemeinen Stand des Wissens erwerben.}\cite{DHBW_Studienarbeit}\\
Am Beginn von LearnAhead wurden uns von der zuständigen Betreuerin Frau Claudia Zinser zwei wissenschaftliche Bücher ausgehändigt, aus welchen wir Informationen über neuronalen Lernens und andere dem Lernen zugeschriebene Gebiete erfassten.\\
Diese Masse an Wissen musste erst einmal neben dem eigentlichen Hochschulbetrieb verarbeitet werden. Dies zog sich bis in die Mitte der 5. Praxisphase, da nicht nur die beiden ausgehändigten Bücher gelesen wurden, sondern auch andere Dokumente Teil unserer Literaturrecherche waren.\\
Bei der kompletten Durchführung der Studienarbeit war es so, dass diese immer einen kleinen Zeitslot benötigte, egal wann, da es sich hierbei ja um ein kontinuierliches Projekt handelte.\\
Es war interessant und äußerst lehrreich ein Projekt zu erleben, welches von Grundauf, von Anfang bis Ende geplant und implementiert wurde.\\
Wir sind mit dem Ablauf der Studienarbeit sehr zufrieden und haben es eingesehen, dass wir an manchen Stellen bei der Planung überambitioniert waren, jedoch freuen wir uns, das Projekt an kommende Generationen von Studenten weiterzugeben und das Projekt wachsen zu sehen.\\
LearnAhead war kein Projekt, was mit bereits vorhandenem Wissen aus dem Boden gestampft wurde, es wurden uns an vielen Stellen über den Prozess des Lernens, nach nun mehr als fast 15 Jahren Lernen in allen möglichen Schulen, die Augen geöffnet.\\ 
Abseits der Hochschule haben wir nun ein umfassendes Interesse an dem neurologischen Prozess des Lernens, da sich hierfür auch sehr viele Erkenntnisse über maschinelles Lernen ziehen lassen. \\
\subsection{Ausblick und die Zukunft}
Mit der Veröffentlichung von LearnAhead geht ein kleines Kapitel in unserer Geschichte zu Ende, aber ein anderes Tor wird noch viel weiter aufgestoßen, welches in die Zukunft von LearnAhead führt.\\
Viele Features könnten noch in LearnAhead implementiert werden, im nachfolgenden Abschnitt werden einige davon aufgelistet.\\
\begin{enumerate}
    \item Lerninhalte, sowie dessen Tests und Fragen müssen momentan noch manuell eingegeben werden. Dies hat zwar den Vorteil, dass hierdurch die Verbindungen im Gehirn des Lernenden gefestigt werden.\\
    Für die Zukunft gäbe es noch die Möglichkeit die App in so einer Form zu erweitern, dass Lerninhalte und Tests oder ganze Lernkategorien importiert, beziehungsweise exportiert werden können, so dass ein neuer Jahrgang den vergangen Jahrgang nach der Lernkategorie für die anstehende Klausur fragen könnte.
    \item Ein anderes Team aus Studierenden haben im Rahmen ihrer Studienarbeit eine Lern-App entwickelt, welche ihren Fokus auf kollaboratives Lernen legt. In Absprache mit der zuständigen Betreuerin Claudia Zinser könnten hierbei beide Lern-Apps über eine API oder Ähnliches miteinander verbunden werden.
    \item Bei den Zusammenfassungen ist es aktuell noch mühsam die Inhalte selber von Grund auf einzugeben. Um dem entgegenzuwirken könnte ein Template-System aufgesetzt werden, welches es den Nutzern ermöglicht Vorlagen für Zusammenfassungen zu erstellen.
    \item Bisher wird dem User das Lernen nur durch einige zufällig ausgewählte Lerntipps beigebracht. Unsere Vision am Anfang von LearnAhead war ein Bereich der App, in welchem man das Lernen lernen kann und allgemeine Falschannahmen zum lernen aufgeklärt werden.
    \item Der User wird dafür belohnt, wenn er eine Frage nach mehrmaligem falsch beantworten richtig beantwortet. Es könnte zur Unterstützung des Lernprozesses nun ein Algorithmus implementiert werden, welcher den User nach mehrmaligem falsch beantworten einer Frage auf die zugehörige Zusammenfassung hinleitet.\\
    Hierfür müssten Zusammenfassungsinhalte mit Fragen verknüpft werden. Dies könnte beispielsweise über die Tag-Funktion geschehen, indem man einen Tag die Zugehörigkeit zu einer Zusammenfassung zuspricht.
    \item Der Markdown-Editor unterstützt bisher nur Basic-Markdown Syntax. Markwon stellt die Möglichkeit zur Verfügung Plugins einzubauen, welche diese Syntax erweitern. Es könnte eine Funktion zum einfügen von mathematischen Operationen implementiert werden, wie es MathJax (\href{URLhttps://www.mathjax.org/}{URLhttps://www.mathjax.org/}) in LaTeX implementiert.
\end{enumerate}


  %%%%%%%%%%%%%%%%%%%%%%% Literaturverzeichnis %%%%%%%%%%%%%%%%%%%%%%%
  \printbibliography[title=Referenzen]
\addcontentsline{toc}{section}{Referenzen}
\newpage


  %%%%%%%%%%%%%%%%%%%%%%%%%%%%%% Anhang %%%%%%%%%%%%%%%%%%%%%%%%%%%%%%
  \renewcommand{\thetable}{\Alph{section}.\arabic{table}}
  \renewcommand{\thefigure}{\Alph{section}.\arabic{figure}}
  \renewcommand{\thelstlisting}{\Alph{section}.\arabic{lstlisting}}
  \pagenumbering{Alph}

  \begin{appendix}
  \section{Anhang}




\end{appendix}
\end{document}