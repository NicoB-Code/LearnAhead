%%%%%%%%%%%%%%%%%%%%%%%%%%%%%%%%%%%%%%%%%%%%%%%%%%%%%%%%%%%%%%%%%%%%
%%%           Vorlage für eine Ausarbeitung an der DHBW          %%%
%%%                                                              %%%
%%%      Bereiche die bearbeitet werden müssen werden durch      %%%
%%%      einen solchen Kommentarblock eingeleitet und enden      %%%
%%%      mit der nächsten Trennlinie.                            %%%
%%%                                                              %%%
%%%      In dieser Datei müssen folgende Bereiche bearbeitet     %%%
%%%      werden:                                                 %%%
%%%      - Angaben zur Arbeit                                    %%%
%%%      - EIGENE KAPITEL EINFÜGEN                               %%%
%%%                                                              %%%
%%%      Benötigte Seiten und Verzeichnisse können unter         %%%
%%%      "Einführung und Verzeichnisse" ein- bzw. auskommentiert %%%
%%%      werden.                                                 %%%
%%%                                                              %%%
%%%%%%%%%%%%%%%%%%%%%%%%%%%%%%%%%%%%%%%%%%%%%%%%%%%%%%%%%%%%%%%%%%%%

\documentclass[a4paper,12pt]{article}
\usepackage[left=2.5cm,right=2.5cm,top=2.5cm,bottom=2.5cm,includehead]{geometry}      % Einstellungen der Seitenränder
\usepackage[english, ngerman]{babel}                                                  % deutsche Silbentrennung
\usepackage[utf8]{inputenc}                                                           % Umlaute
\usepackage[T1]{fontenc}													                                    % Umlaute auch richtig ausgeben
\usepackage{newtxtext,newtxmath}                                                      % Font = Times New Roman
\usepackage{hyperref}
%\usepackage[nottoc]{tocbibind}
\usepackage{fancyhdr}
\usepackage{setspace}
\usepackage[backend=bibtex, style=numeric,sortcites,sorting=nty,backref,natbib,hyperref]{biblatex}      % Bibliothek für Zitate
\usepackage{csquotes}                                                                 % Zusatzpacket für Zitate
\usepackage{amsmath}                                                                  % Zurücksetzen der Tabellen- und Abbildungsnummerierung je Sektion
\usepackage[labelfont=bf]{caption}                                      % Bild- und Tabellenunterschrift (fett)
\usepackage[bottom,multiple,hang,marginal]{footmisc}                                  % Fußnoten [Ausrichtung unten, Trennung durch Seperator bei mehreren Fußnoten]
\usepackage{graphicx}  
\graphicspath{{./images/}}                                                            % Grafiken
\usepackage[dvipsnames]{xcolor}                                                       % Farbige Buchstaben
\usepackage{wrapfig}                                                                  % Bilder in Text integrieren
\usepackage{enumitem}                                                                 % Befehl setlist (Zeilenabstand für itemize Umgebung auf 1 setzen)
\usepackage{listings, chngcntr}                                                                 % Quelltexte
\usepackage{subcaption}
\definecolor{lightgray}{rgb}{.9,.9,.9}
\definecolor{darkgray}{rgb}{.4,.4,.4}
\definecolor{purple}{rgb}{0.65, 0.12, 0.82}
\lstdefinelanguage{JavaScript}{
	keywords={typeof, new, true, false, catch, function, return, null, catch, switch, var, if, in, while, do, else, case, break,let},
	ndkeywords={class, export, boolean, throw, implements, import, this},
	comment=[l]{//},
	morecomment=[s]{/*}{*/},
	morestring=[b]',
	morestring=[b]"
}
\lstset{
	keywordstyle=\color{blue}\bfseries,
	ndkeywordstyle=\color{darkgray}\bfseries,
	identifierstyle=\color{black},
	sensitive=false,
	commentstyle=\color{purple}\ttfamily,
	stringstyle=\color{red}\ttfamily,
	backgroundcolor=\color{lightgray},
	extendedchars=true,
	basicstyle=\footnotesize\ttfamily,
	showstringspaces=false,
	showspaces=false,
	numbers=left,
	numberstyle=\footnotesize,
	numbersep=9pt,
	tabsize=2,
	breaklines=true,
	showtabs=false,
	captionpos=b
}
\usepackage{tabularx}                                                                 % Tabellen
\usepackage[nohyperlinks, printonlyused, withpage]{acronym}                           % Abkürzungen
\usepackage{dirtree}      
\usepackage{xcolor}
\usepackage{float}
\usepackage{rotating}
\usepackage{tocbibind}	% Inhaltsverzeichnis Eintrag im Inhaltsverzeichnis
\usepackage{tikz}
%\usepackage{svg} % Einbinden von Vektorgrafiken in LaTeX

\usetikzlibrary{positioning, shapes, calc}
%%%%%%%%%%%%%%%%%%%%%%%%%%%%%%%%%%%%%%%%%%%%%%%%%%%%%%%%%%%%%%%%%%%%
%%%                      Angaben zur Arbeit                      %%%
%%%%%%%%%%%%%%%%%%%%%%%%%%%%%%%%%%%%%%%%%%%%%%%%%%%%%%%%%%%%%%%%%%%%
\def\vFirmenlogoPfad{}                        %% relativer Pfad Bsp.: images/Firmenlogo.png
\def\vDHBWLogoPfad{images/DHBW\_logo.jpg}                          %% relativer Pfad Bsp.: images/DHBW_logo.jpg

\def\vTitel{Entwicklung einer mobilen Lern-Applikation}                           %% 
\def\vUntertitel{}                      %% 
\def\vArbeitstyp{Studienarbeit}                      %% Projektarbeit/Seminararbeit/Bachelorarbeit

\def\vAutor{Phillipp Patzelt, Nico Bayer}                           %% Vorname Nachname
\def\vMatrikelnummer{8138164. 3056312}                  %% 7-stellige Zahl
\def\vKursKuerzel{TIT20}                     %% Bsp.: TIT20
\def\vPhasenbezeichnung{Theoriephase}               %% Praxisphase/Theoriephase
\def\vStudienJahr{dritte}                     %% erste/zweite/dritte
\def\vDHBWStandort{Ravensburg}                    %% Bsp.: Ravensburg
\def\vDHBWCampus{Friedrichshafen}                      %% Bsp.: Friedrichshafen
\def\vFakultaet{Technik}                       %% Technik/Wirtschaft
\def\vStudiengang{Informationstechnik}                     %% Informationstechnik/...

\def\vBetrieb{}                         %% 
\def\vBearbeitungsort{}                 %% 
\def\vAbteilung{}                       %% 
\def\vBetreuer{Claudia Zinser}                        %% Vorname Nachname

\def\vAbgabedatum{Frau Schmidt Mail nachfragen}               %% DD. MONTH YYYY
\def\vBearbeitungszeitraum{Nochmal überprüfen}            %% DD.MM.YYYY - DD.MM.YYYY


%%%%%%%%%%%%%%%%%%%%%%%%% Eigene Kommandos %%%%%%%%%%%%%%%%%%%%%%%%%
% Definition von \gqq{}: Text in Anführungszeichen
\newcommand{\gqq}[1]{\glqq #1\grqq}
% Spezielle Hervorhebung von Schlüsselwörtern
\newcommand{\textOrdner}[1]{\texttt{#1}}
\newcommand{\textVariable}[1]{\texttt{#1}}
\newcommand{\textKlasse}[1]{\texttt{#1}}
\newcommand{\textFunktion}[1]{\texttt{#1}}


%%%%%%%%%%%%%%%%%%%% Zitatbibliothek einbinden %%%%%%%%%%%%%%%%%%%%%
\addbibresource{./literatur/literatur.bib}


%%%%%%%%%%%%%%%%%%%%%%%% PDF-Einstellungen %%%%%%%%%%%%%%%%%%%%%%%%%
\hypersetup{
	bookmarksopen=false,
	bookmarksnumbered=true,
	bookmarksopenlevel=0,
	pdftitle=\vTitel,
	pdfsubject=\vTitel,
	pdfauthor=\vAutor,
	pdfborder={0 0 0},
	pdfstartview=Fit,
	pdfpagelayout=SinglePage
}

%%%%%%%%%%%%%%%%%%%%%%%% Kopf- und Fußzeile %%%%%%%%%%%%%%%%%%%%%%%%
\pagestyle{fancy}
\setlength{\headheight}{15pt}
\fancyhf{}
\fancyhead[R]{\thepage}


%%%%%%%%%%%%%%%%%%%%%%%%%%%%%% Layout %%%%%%%%%%%%%%%%%%%%%%%%%%%%%%
\onehalfspacing
\setlist{noitemsep}
\widowpenalties=3 10000 10000 150	% Umbrüche

\addto\captionsngerman{
  \renewcommand{\figurename}{Abb.}
  \renewcommand{\tablename}{Tab.}
}
\renewcommand{\thetable}{\arabic{section}.\arabic{table}}   % Tabellennummerierung mit Section
\renewcommand{\thefigure}{\arabic{section}.\arabic{figure}} % Abbildungsnummerierung mit Section
\renewcommand{\thefootnote}{\arabic{footnote}}              % Sektionsbezeichnung von Fußnoten entfernen

\renewcommand{\multfootsep}{, }                             % Mehrere Fußnoten durch ", " trennen


%%%%%%%%%%%%%%%%%%%%%%%%%%%%% Dokument %%%%%%%%%%%%%%%%%%%%%%%%%%%%%

\begin{document}
\counterwithin{lstlisting}{subsection} 						% Muss nach \begin{document} stehen, da ansonsten nicht bekannt
\counterwithin{table}{subsection}                               % Tabellennummerierung je Sektion zurücksetzen
\counterwithin{figure}{subsection}                              % Abbildungsnummerierung je Sektion zurücksetzen
\newsavebox{\savefig}

  %%%%%%%%%%%%%%%%%%% Einführung und Verzeichnisse %%%%%%%%%%%%%%%%%%%
  \pagenumbering{Roman}

  \begin{titlepage}
  \begin{minipage}{6in}
    \vspace*{-2cm}
    \centering
    \hspace{-2cm}
	\ifx\vFirmenlogoPfad\empty
	\else
    \raisebox{-0.5\height}{\includegraphics[height=3cm]{\vFirmenlogoPfad}}
  \fi
	\hfill
	\ifx\vDHBWLogoPfad\empty
	\else
   	\raisebox{-0.5\height}{\includegraphics[height=3cm]{\vDHBWLogoPfad}}
	\fi
  \end{minipage}
  \begin{center}
    \vspace*{0.5cm}
    \Huge\textbf{\vTitel}\\
		\ifx\vUntertitel\empty
		\else
			\Large\rm\vUntertitel\\
		\fi
		\vspace*{2cm}
		\textbf{\vArbeitstyp}\\
		\normalsize
		\vspace*{1.3cm}
		an der Fakultät für \vFakultaet\\
		im Studiengang \vStudiengang\\
		\vspace*{1cm}
		an der \ac{DHBW} \vDHBWStandort\\
		\ifx\vDHBWCampus\empty
		\else
		Campus \vDHBWCampus\\
		\fi
		\vspace*{1cm}
		von\\
		\ifx\vAutor\empty
		\else
			\vAutor\\
		\fi
		\vspace*{2cm}
		\vfill
  
	  \begin{tabular}{ll}
	    Bearbeitungszeitraum:&\vBearbeitungszeitraum\\
	    Matrikelnummer, Kurs:&\vMatrikelnummer, \vKursKuerzel\\
		  Betreuerin der Studienarbeit:&\vBetreuer\\
	  \end{tabular}
	\end{center}
\end{titlepage}
\newpage
\setcounter{page}{2}
  % \include{pages/sperrvermerk}
  \section*{\Huge{Selbstständigkeitserklärung}}
gemäß Ziffer 1.1.13 der Anlage 1 zu §§ 3, 4 und 5  der Studien- und Prüfungsordnung für die Bachelorstudiengänge im Studienbereich Technik der Dualen Hochschule Baden-Würt­tem­berg vom 29.09.2017.

\noindent Hiermit versicheren wir, dass wir unsere Studienarbeit mit dem Thema: 
\begin{center}
	\Large\textbf{\vTitel}
\end{center}
selbstständig verfasst und keine anderen als die angegebenen Quellen und Hilfsmittel benutzt haben. Wir versichern zudem, dass die eingereichte elektronische Fassung mit der gedruckten Fassung übereinstimmt.

\vfill
\leavevmode
\newline
\parbox{6cm}{\hrule\strut\centering\footnotesize Ort, Datum} 
\hfill
\parbox{6cm}{\hspace{1pt} \vAbteilung\hrule\strut\centering\footnotesize Unterschriften Studierende }

\newpage
  \begin{center} \section*{Abstract} \end{center} 
\newenvironment{keywords}{
	\begin{flushleft}
	\small	
	\textbf{
		\iflanguage{ngerman}{Schlüsselwörter}{\iflanguage{english}{Keywords}{}}
	}
}{\end{flushleft}}

% Deutsche Zusammenfassung
\begin{abstract}\noindent 
	
\end{abstract}

% Schlüsselwörter Deutsch
\begin{keywords}

\end{keywords}


\selectlanguage{english}
% Englisches Abstract
\begin{abstract}\noindent 
\end{abstract}
% Schlüsselwörter Englisch
\begin{keywords}

\end{keywords}


\selectlanguage{ngerman}
\newpage
  \makeatletter
\renewcommand{\tableofcontents}{\@starttoc{toc}}
\makeatother
\subpdfbookmark{Inhaltsverzeichnis}{Inhaltsverzeichnis}
\section*{Inhaltsverzeichnis}
\tableofcontents
  \section*{Abkürzungsverzeichnis}
\addcontentsline{toc}{section}{Abkürzungsverzeichnis}
\begin{acronym}
  \acro{DHBW}{Duale Hochschule Ba\-den-\-Würt\-tem\-berg}  
  \acro{MVVM}{Model View ViewModel}
  \acro{DSGVO}{Datenschutz Grundverordnung}
  \acro{App}{Applikation}
  \acro{LTP}{Long-Term Potentiation}
  \acro{LTD}{Long-Term Depression}
  \acro{DI}{Dependency Injection}
  \acro{UI}{User Interface}
\end{acronym}
\newpage
  \include{pages/abbildungsverzeichnis}
  \include{pages/tabellenverzeichnis}
  % \include{pages/listingsverzeichnis}
  % \include{pages/vorwort}


  %%%%%%%%%%%%%%%%%%%%%%%%%%%%% Kapitel %%%%%%%%%%%%%%%%%%%%%%%%%%%%%%
  \pagestyle{fancy}
  \fancyhead[L]{\nouppercase{\rightmark}}    % Abschnittsname im Header
  \pagenumbering{arabic}

  %%%%%%%%%%%%%%%%%%%%%%%%%%%%%%%%%%%%%%%%%%%%%%%%%%%%%%%%%%%%%%%%%%%%
  %%%%                   EIGENE KAPITEL EINFÜGEN                  %%%%
  %%%%%%%%%%%%%%%%%%%%%%%%%%%%%%%%%%%%%%%%%%%%%%%%%%%%%%%%%%%%%%%%%%%%
  \section{Einleitung}
\gqq{Es ist keine Schande nichts zu wissen, wohl aber nichts lernen zu wollen.}\\
Diese Worte verwendete schon der griechische Philosoph Platon. Wir hören nie auf zu lernen, das ist ein Fakt. \\
Der Prozess des Lernens lässt sich nicht aktiv unterbinden. Lernen ist der Impuls, Informationen zu verarbeiten, einzuordnen und zu verstehen. Das Behalten des Erlernten passiert ebenfalls automatisch, wenn auch bei vielen Dingen nicht dauerhaft. Der Mensch ist so angelegt, dass er seine Umwelt begreifen will. Somit ist Lernen ein intrinsisch motivierter Prozess, der zwar angeregt, aber nicht verordnet werden kann.
Etwa 100 Milliarden Nervenzellen besitzt unser Gehirn, die sogenannten Neuronen. An deren Ende liegen Synapsen, die elektrische Signale der Neuronen in Form von chemischen Botenstoffen an die nächsten Neuronen abgegeben. Diese Kettenreaktionen tragen die Informationen und Signale durch unser neuronales Netz im Gehirn an die Stellen, wo es benötigt wird, egal ob wir unsere Muskeln bewegen wollen oder unsere Sinne einsetzen, und lösen dort Reaktionen aus. \\
Obwohl der Mensch also nicht nur in der Theorie, sondern auch in der Praxis für das Lernen konstruiert ist, macht das Lernen vielen Menschen nicht spaß und sie behaupten: \gqq{Sie können nicht lernen}. \\
Faktisch gesehen ist das jedoch nicht richtig! Jahrhunderte, ja sogar Jahrtausende der Evolution haben uns dazu ausgebildet nie aufzuhören zu lernen. Die heutige Welt bietet jedoch viele Tücken, welche das Lernen einschränken können. \\
Das Lernen ist ein komplexer, jedoch kein komplizierter Prozess, es muss nicht der komplette Bio-chemische Ablauf verstanden werden, jedoch sollte man einige wichtige Grundlagen beachten.
  \section{Planungsphase}
\label{Planung}

\subsection{Lastenheft}

\subsubsection{Allgemeine Beschreibung}
Eine mobile LernApp, mit der Möglichkeit Lerninhalte zusammenzufassen und von diesen in Abhängigkeit Tests zu erstellen.
Von den eingetragenen Lerninhalten können dann Tests erstellt werden, welche aus verschiedenen Fragen bestehen.
Es besteht die Möglichkeit die Inhalt nach Kategorien zu gruppieren und Lernzielen zu zuweisen.
Mit diesen Lernzielen wird dem User dann ein empfohlener Lernplan erstellt.

\subsubsection{Anforderungen}
\begin{itemize}
    \item Lerninhalte zusammenfassen
    \item Tests erstellen
    \item Lernkategorien
    \item Auf Android im Play Store verfügbar
    \item Lernziele und dazugehörigen automatisch erstellten Lernplan
    \item User-Profil mit Benutzername und Kennwort
\end{itemize}

\subsubsection{Fachliches Umfeld}
\begin{itemize}
    \item Plattformabhängig mit Android Studio
    \item Mobile Lösung
    \item Datenbank
    \item IT-Security
    \item DSGVO
\end{itemize}

\subsubsection{Ausblick}
Bei erfolgreichen Entwicklungsergebnissen soll die Lösung in Betrieb genommen und der Öffentlichkeit, per Play Store, zur Verfügung gestellt werden.

\subsubsection{Erweiterungsmöglichkeiten (optional)}
\begin{itemize}
    \item Importieren und Exportieren von Lerninhalten auf WhatsApp oder Ähnlichem
\end{itemize}


\subsection{Arbeitspaketplan}
\label{sec:arbeitspaketplan}
Der Arbeitspaketplan bezeichnet die Aufzählung jedes Arbeitspakets auf Basis des Lastenhefts. \newline Ein Arbeitspaket wird durch folgendes definiert: 
\begin{itemize}
    \item Ein definiertes Ergebnis (was soll in diesem Arbeitspaket erreicht werden)
    \item Der zeitliche Aufwand des Arbeitspakets
    \item Die Vorbedingungen, die beim Bearbeiten zu beachten sind
    \item Die Dauer
\end{itemize}


\noindent
Um die Arbeitspakete grafisch aufbereitet darstellen zu können, werden diese in die Anwendung \href{https://studienarbeitlernapp.atlassian.net/jira/software/projects/LER/boards/1}{\underline{Jira}}\footnote{\href{https://studienarbeitlernapp.atlassian.net/jira/software/projects/LER/boards/1}{https://studienarbeitlernapp.atlassian.net/jira/software/projects/LER/boards/1}} ausgelagert.
Hier wird der Arbeitspaketplan als Unterteilung der einzelnen Epics in User Stories dargestellt.
Erledigte Epics und User Stories sind dabei unter dem Reiter \gqq{Fertig} einsehbar.
In den einzelnen User Stories wird ein definiertes Ergebnis aus Sicht des Nutzers beschrieben.
Der zeitliche Aufwand der einzelnen Arbeitspakete ergibt sich aus der Summe der Dauer der zugewiesenen Tasks. 
In Kombination mit Scrum werden dabei vor Beginn des jeweiligen Sprints die einzelnen Tasks geschätzt und auf einen festen Arbeitsaufwand festgelegt. 

Im Laufe des Sprints werden dann die zugewiesenen Stunden von den zugewiesenen Bearbeitern abgebaut und im jeweiligen Task aktualisiert.
Ist der Task beendet, so wird er mit \gqq{Done} markiert und besitzt somit keinen Arbeitsaufwand mehr.

Vorbedingungen, sowie die festgelegte Dauer für die Bearbeitung eines Arbeitspaketes werden durch die Sprints definiert. 
Durch Aufteilen der Tasks in verschiedene, nacheinander ablaufende Sprints, können Vorbedingungen durch Einteilung in einen vorangehenden Sprint gesetzt werden.
Darüber hinaus können den einzelnen Tasks Prioritäten zwischen eins und vier zugeordnet werden, was eine Hierarchie innerhalb eines Sprints ermöglicht.
Zusätzlich limitiert die Dauer des Sprints die Bearbeitungszeit für den jeweiligen Task.
\subsection{Zeitplan}
\subsection{Qualitätsmanagement Maßnahmen}
\subsection{Konfigurationsmanagement Maßnahmen}
Die agile Planung im erweitertem Scrum erfolgt in Jira, hier ist der Backlog angelegt, in welchem die Sprint-Planung erfolgt. Meetings werden auf Discord durchgeführt. Die Dokumentation wird mit \LaTeX  geschrieben.\newline
Die Versions- und Releaseverwaltung erfolgt in einem GitHub Repository unter dem Git-Branching-Modell Gitflow. Gitflow sieht zwei Branches vor um den Verlauf des Projekts aufzuzeichnen. Der main-Branch enthält den offiziellen Release-Verlauf und der develop-Branch dient als Integrations-Branch für Features. Der develop-Branch enthält den kompletten Versionsverlauf des Projekts, während der main-Branch eine verkürzte Version enthält.\newline
Jedes neue Feature sollte sich auf seinem eigenen Branch befinden, der zu Sicherungs-/Zusammenarbeitszwecken zum zentralen Repository gepusht werden kann. Ein neuer feature-Branch  wird aus dem develop-branch gemerget. Wenn ein Feature fertig ist, wird es zurück in den develop-Branch gemergt. Features sollten niemals direkt mit dem main-Branch interagieren.\newline
Wenn der develop-Branch genügend Features für ein Release enthält, wird vom develop-Branch ein release-Branch geforkt. Damit beginnt der neue Release-Zyklus. In diesem Branch sollten ab diesem Punkt keine neuen Features mehr hinzugefügt werden, sondern nur Bugfixes und ähnliche Release-orientierte Änderungen. Ist er zur Auslieferung bereit, wird der release-Branch in den main-Branch gemergt und mit einer Versionsnummer getaggt. Zusätzlich wird der release-Branch in den develop-Branch gemerged. \newline
Maintenance- bzw. hotfix-Branches eignen sich für schnelle Patches von Produktions-Releases. Sie werden aus dem main-Branch geforkt. Sobald der Bugfix abgeschlossen ist, wird er sowohl in den main- als auch in den develop-Branch (oder den aktuellen release-Branch) gemergt.
\subsection{Auswahl kritischer Technologien}





  \section{Definition}
\subsection{Definition technischer Begriffe}
\subsection{Pflichtenheft}
\subsection{Use-Case-Diagramme}
\subsection{Use-Case-Beschreibungen}
\subsection{Datenbankmodell}
\begin{figure}[H]
    \centering
    \includegraphics[width=1\textwidth]{images/LernAhead Datenbankstruktur.png}
    \caption{LearnAhead Datenbankstruktur}
    \label{fig:LearnAheadDatenbankstruktur}
\end{figure}\noindent
\subsection{Data Dictionary}
\subsection{HMI}
\subsubsection{Seitenhierarchie}
\subsubsection{UI-Mockups}
\subsection{Datenflussdiagramm}
\subsection{Benutzerhandbuch}
  \section{Entwurf}
\subsection{Geschäftsfälle anhand des BPMN Workflows}
\subsection{Auswahl und Begründung des Datenbankkonzepts}
Die Lernapp \gqq{LearnAhead} benötigt eine Datenbank, um die Lerninhalte des Benutzers zu speichern. Die Datenbank soll die folgenden Anforderungen erfüllen:
\begin{itemize}
    \item Es soll möglich sein, Bilder einfach zu speichern und abzurufen.
    \item Die Datenbank soll gut skalierbar sein, um auch bei vielen Benutzern eine gute Performance zu gewährleisten.
    \item Es müssen gute Frameworks für die Anbindung existieren für die Sprache Kotlin.
    \item Die Datenbank soll in der Cloud gehostet werden, um die Wartungskosten zu minimieren.
    \item Die Datenbank soll möglichst kostenlos sein.
\end{itemize}

\noindent
Hierbei gab es die Entscheidung zwischen einer relationalen und einer NoSQL Datenbank. Da diese für die Speicherung von Bildern zuständig ist, ist eine NoSQL Datenbank die bessere Wahl. In diesem Zusammenhang kam der Anbieter \gqq{Firebase} in Frage. Dieser bietet zwei unterschiedliche Datenbanken an: \gqq{Cloud Firestore} und \gqq{Realtime Database}. Realtime Database wird genutzt, wenn die Datenbank in Echtzeit synchronisiert werden soll. Da dies bei der Lernapp nicht notwendig ist, wurde sich für Cloud Firestore entschieden. Cloud Firestore baut auf den Erfolgen von Realtime Database auf und bietet zusätzlich eine bessere Skalierbarkeit und schnellere Abfragen sind möglich. \cite*{Firestore} Für die Speicherung von Inhalten gibt es den Firebase Storage. Dieser bietet eine einfache Möglichkeit, um Inhalte z.B. Bilder oder Videos zu speichern. \newline

\noindent
Firebase bietet eine gute Anbindung für Kotlin siehe \ref*{Auswahl der Klassenbibliotheken/Frameworks}. Zusätzlich ist Firebase kostenlos, solange die Datenbank nicht zu groß wird. Hierbei kann eine monatliche Anzahl von 50.000 Nutzern, 20.000 Schreibvorgängen und 50.000 Lesevorgängen kostenlos genutzt werden. Der Speicherplatz für den Firebase Storage beträgt 1 GB, welches für die Lernapp ausreichend ist. \cite*{Firebase_Pricing}
\subsection{Auswahl der Klassenbibliotheken/Frameworks} \label{Auswahl der Klassenbibliotheken/Frameworks}
\subsection{Design Patterns für relevante Problemstellungen} \label{Design Patterns für relevante Problemstellungen}
\subsection{Software-Komponenten}
Das System wird zunächst als Gesamtkomposition dargestellt. Anschließend werden die Subsysteme einzeln dargestellt. 
\subsubsection{Gesamtkomposition}
\begin{figure}[H]
    \centering
    \includegraphics[width=0.8\textwidth]{images/diagramme/Gesamtkomposition.png}
    \caption{Gesamtkomposition}
    \label{fig:Gesamtkomposition}
\end{figure}
\noindent
Die Gesamtkomposition besteht aus den folgenden Subsystemen:
\begin{itemize}
    \item \textbf{Frontend:} Das Frontend ist für die Darstellung der Benutzeroberfläche zuständig. Es kommuniziert mit dem Backend, um Daten abzurufen und zu speichern.
    \item \textbf{Backend:} Das Backend ist für die Verarbeitung und Validierung der Daten zuständig. Es kommuniziert mit der FireStore Datenbank, um Daten abzurufen und zu speichern.
    \item \textbf{FireStoreDB} Die FireStoreDB Komponente ist die Datenbank-Komponente.
\end{itemize}

\noindent
Näheres hierzu im Deployment Diagramm \ref*{Deployment Diagramm} und Klassen Diagramm \ref*{Klassen Diagramm}.
\newpage
\subsubsection{Frontend}
Es gibt zwei Diagramme für das Frontend. Das Diagramm \ref*{fig:FrontendUINavigation} zeigt die Navigation zwischen den einzelnen Pages.
\begin{figure}[H]
    \centering
    \includegraphics[width=0.8\textwidth]{images/diagramme/FrontEndKomposition.png}
    \caption{Frontend UI Navigation}
    \label{fig:FrontendUINavigation}
\end{figure}
\noindent
Das Diagramm \ref*{fig:FrontendUIViewModel} zeigt die Kommunikation zwischen den einzelnen Pages und dem ViewModel. Hierbei ist zu beachten, dass dieses Diagramm die Kommunikation zwischen den einzelnen Pages und dem ViewModel im allgemeinen zeigt. Das bedeutet, dass jede Page ein zugehöriges ViewModel hat. Die Kommunikation zwischen den einzelnen Pages und dem ViewModel ist prinzipiell immer gleich. Weitere Details sind in Kapitel \ref*{Design Patterns für relevante Problemstellungen} zu finden. \newline
\begin{figure}[H]
    \centering
    \includegraphics[width=0.8\textwidth]{images/diagramme/UIPagesMitViewModel.png}
    \caption{Frontend UI Kommunikation mit ViewModel}
    \label{fig:FrontendUIViewModel}
\end{figure}
\subsubsection{Backend}
Das Diagramm \ref*{fig:BackendKomposition} zeigt die Komposition des Backends. Hierbei ist zu beachten, dass dieses Diagramm die Komposition des Backends im allgemeinen zeigt. Jedes Objekt, welches in der Datenbank gespeichert wird, hat ein zugehöriges Repository/Model. Die Kommunikation zwischen den verschiedenen Repositories und Modellen verläuft grundsätzlich immer auf die gleiche Weise. Jedes Repository baut auf ein Interface auf, welches die grundlegenden Funktionen für die Kommunikation mit der Datenbank definiert. Die Kommunikation mit der Datenbank folgt über das Repository.
\begin{figure}[H]
    \centering
    \includegraphics[width=0.8\textwidth]{images/diagramme/Backend.png}
    \caption{Backend Komposition}
    \label{fig:BackendKomposition}
\end{figure}
\subsection{Deployment Diagramm} \label{Deployment Diagramm}
\subsection{Klassen Diagramm} \label{Klassen Diagramm}
\subsection{Aktivitätsdiagramm}
\subsection{Sequenzdiagramme}
\subsection{Prototyp (optional)}
  \section{Implementierung}
In diesem Kapitel wird die komplette Implementierung von LearnAhead erläutert. Es wird beispielsweise explizit darauf eingegangen wie die App im Google Play Store deployed wurde und wie, aus Sicht eines Users sowie eines Entwicklers zu installieren ist.\newline

\subsection{Deployment der LearnAhead App im Google Play Store}
Die Veröffentlichung der LearnAhead App im Google Play Store bestand aus mehreren Schritten, die im Folgenden detailliert beschrieben werden. \newline
\subsubsection{Vorbereitung der App für die Veröffentlichung}
Bevor die App im Google Play Store veröffentlicht werden konnte, musste sie für die Veröffentlichung vorbereitet werden. Dies beinhaltete die Überprüfung und Anpassung der App-Manifest-Datei, das Bereinigen von Debug-Code und Log-Ausgaben, das Aktualisieren der App-Ressourcen und das Testen der App in verschiedenen Gerätekonfigurationen.\newline
\subsubsection{Erstellen einer signierten APK oder AAB}
Nach der Vorbereitung der App wurde eine signierte APK (Android Package) oder AAB (Android App Bundle) erstellt. Dies ist die Datei, die im Google Play Store hochgeladen wurde. Ein privates Schlüsselpaar wurde erstellt, um die App zu signieren. \newline
\subsubsection{Erstellen eines Google Play Entwicklerkontos}
Um die App im Google Play Store veröffentlichen zu können, wurde ein Google Play Entwicklerkonto erstellt. Es wurde eine einmalige Anmeldegebühr von 25 USD gezahlt. \newline
\subsubsection{Erstellen einer neuen App im Play Console}
Nach der Erstellung des Entwicklerkontos wurde eine neue App in der Google Play Console erstellt. Informationen wie der Name der App, die Beschreibung, die Kategorie und die Altersfreigabe wurden angegeben.\newline
\subsubsection{Hochladen der APK und Bereitstellung von Store-Listing-Informationen}
Nach der Erstellung der App wurde die APK hochgeladen und weitere Informationen für das Store-Listing bereitgestellt, wie Screenshots, App-Symbole, Werbetexte und Kontaktinformationen.\newline
\subsubsection{Festlegen der Preisgestaltung und Verteilung}
Es wurde entschieden, dass die App kostenlos ist und in welchen Ländern sie verfügbar sein soll.\newline
\subsubsection{Veröffentlichung der App}
Nachdem alle erforderlichen Informationen bereitgestellt wurden, wurde die App zur Überprüfung eingereicht. Nach der Genehmigung durch Google wird die LearnAhead App im Google Play Store veröffentlicht.

\subsection{Installationsanleitung}
In diesem Kapitel wird erläutert, wie die LearnAhead App sowohl von Endbenutzern als auch von Entwicklern installiert werden kann.
\subsubsection{Anleitung für User}
Um die LearnAhead App auf einem Android-Gerät zu installieren, befolgen Sie bitte die folgenden Schritte: \newline
\begin{enumerate}
    \item Öffnen Sie die Google Play Store App auf Ihrem Android-Gerät.
    \item Tippen Sie in der Suchleiste \gqq{LearnAhead} ein und drücken Sie auf die Suchtaste.
    \item Wählen Sie die LearnAhead App aus der Liste der Suchergebnisse aus.
    \item Tippen Sie auf die Schaltfläche \gqq{Installieren}, um den Download und die Installation der App zu starten.
    \item Nach Abschluss der Installation können Sie die App öffnen und verwenden, indem Sie auf die Schaltfläche \gqq{Öffnen} tippen oder das App-Symbol auf Ihrem Gerät suchen und auswählen.
\end{enumerate}
\subsubsection{Anleitung für Developer}
Um die LearnAhead App für Entwicklungs- und Testzwecke zu installieren, befolgen Sie bitte die folgenden Schritte:\newline
\begin{enumerate}
    \item Laden Sie und installieren Sie Android Studio auf Ihrem Computer, falls Sie dies noch nicht getan haben. Sie können Android Studio von der offiziellen Website herunterladen.
    \item Beantragen Sie die Berechtigung, auf das Github-Repository der LearnAhead App zuzugreifen, indem Sie eine E-Mail an bayernico@web.de senden.
    \item Nach Erhalt der Berechtigung, öffnen Sie Android Studio und wählen Sie \gqq{Get from Version Control} aus dem \gqq{File} Menü.
    \item Geben Sie die URL des Github-Repositorys (https://github.com/NicoB-Code/LearnAhead) in das Feld \gqq{URL} ein und klicken Sie auf \gqq{Clone}.
    \item Nachdem das Repository erfolgreich geklont wurde, können Sie die App in Android Studio öffnen, bearbeiten und testen.
\end{enumerate}

\subsubsection{Verwendung des Android-Emulators}
Der Android-Emulator ist ein Tool, das von Android Studio bereitgestellt wird und es Entwicklern ermöglicht, ihre Apps auf einem virtuellen Android-Gerät zu testen. Bitte beachten Sie, dass ihr virtuelles Gerät mindestens Tiramisu (Android 13) benötigt. Um den Android-Emulator zu verwenden, befolgen Sie bitte die folgenden Schritte:\newline
\begin{enumerate}
    \item Starten Sie Android Studio und öffnen Sie Ihr Projekt.
    \item Wählen Sie \gqq{AVD Manager} aus dem \gqq{Tools} Menü.
    \item Klicken Sie auf \gqq{Create Virtual Device}, um ein neues virtuelles Gerät zu erstellen.
    \item Wählen Sie einen Gerätetyp aus der Liste aus und klicken Sie auf \gqq{Next}.
    \item Wählen Sie ein System-Image aus, das Sie auf dem virtuellen Gerät installieren möchten, und klicken Sie auf \gqq{Next}. Wenn Sie noch kein System-Image heruntergeladen haben, können Sie dies tun, indem Sie auf \gqq{Download} neben dem gewünschten Image klicken.
    \item Überprüfen Sie die Konfiguration des virtuellen Geräts und klicken Sie auf \gqq{Finish}, um das virtuelle Gerät zu erstellen.
    \item Um das virtuelle Gerät zu starten, klicken Sie auf das grüne Dreieckssymbol neben dem Gerät in der Liste der verfügbaren virtuellen Geräte im AVD Manager.
    \item Sobald das virtuelle Gerät gestartet ist, können Sie Ihre App darauf ausführen, indem Sie \gqq{Run} aus dem \gqq{Run} Menü wählen und das laufende virtuelle Gerät aus der Liste der verfügbaren Geräte auswählen.
\end{enumerate}

\subsubsection{Übertragen der App zu Entwicklungszwecken ohne Play Store}
Um die App auf einem tatsächlichen Android Gerät nutzen zu können, ohne sie jedes mal in den Google Play Store deployen zu müssen, können gewisse Entwicklertools genutzt werden, welche im nachfolgenden erklärt werden.
\paragraph{Übertragung per USB}
\begin{enumerate}
    \item Verbinden Sie Ihr Android-Gerät über ein USB-Kabel mit Ihrem Computer.
    \item Aktivieren Sie auf Ihrem Android-Gerät die Option \gqq{USB-Debugging} in den Entwicklereinstellungen.
    \item In Android Studio, wählen Sie \gqq{Run} aus dem \gqq{Run} Menü und wählen Sie Ihr angeschlossenes Gerät aus der Liste der verfügbaren Geräte.
    \item Android Studio installiert die App auf Ihrem Gerät und startet sie.
\end{enumerate}

\paragraph{Übertragung per Wi-Fi-Pairing}
\begin{enumerate}
    \item Stellen Sie sicher, dass Ihr Android-Gerät und Ihr Computer mit demselben Wi-Fi-Netzwerk verbunden sind.
    \item Aktivieren Sie auf Ihrem Android-Gerät die Option \gqq{Wireless Debugging} in den Entwicklereinstellungen (hierfür müssen Sie den Entwicklermodus aktiviert haben - siehe https://blog.deinhandy.de/android-entwickleroptionen-aktivieren-so-funktionierts)
    \item In Android Studio, wählen Sie \gqq{Pair Device with QR Code} aus dem \gqq{Run} Menü und scannen Sie den angezeigten QR-Code mit Ihrem Android-Gerät.
    \item Nachdem das Gerät erfolgreich gepaart wurde, können Sie \gqq{Run} aus dem \gqq{Run} Menü wählen und Ihr gepaartes Gerät aus der Liste der verfügbaren Geräte auswählen.
    \item Android Studio installiert die App auf Ihrem Gerät und startet sie.
\end{enumerate}

\subsection{IT-Sicherheit in LearnAhead}
Ein wichtiger Aspekt bei der Entwicklung der LearnAhead App war die Gewährleistung der IT-Sicherheit. Dies wurde durch die Verwendung von Firebase für die Authentifizierung und Datenspeicherung erreicht \cite{firebase_overview}.\newline
\subsubsection{Firebase Authentifizierung}
Firebase Authentifizierung bietet eine sichere und zuverlässige Authentifizierungslösung. Es unterstützt eine Vielzahl von Authentifizierungsmethoden, einschließlich E-Mail und Passwort, und es integriert sich nahtlos mit anderen Firebase-Diensten \cite{firebase_auth}.\newline
Firebase Authentifizierung bietet auch eine Reihe von Sicherheitsfunktionen, um die Konten der Benutzer zu schützen. Dazu gehören unter anderem:
\begin{itemize}
    \item \textbf{Passwortschutz:} Firebase speichert Passwörter sicher und ermöglicht es den Benutzern, ihre Passwörter zurückzusetzen, wenn sie diese vergessen haben \cite{firebase_auth}.
    \item \textbf{Kontosicherheit:} Firebase bietet Schutzmaßnahmen gegen Brute-Force- und DDoS-Angriffe \cite{firebase_security}.
    \item \textbf{Datenschutz:} Firebase hält sich an die Datenschutzbestimmungen und -standards, einschließlich der DSGVO \cite{firebase_privacy}.
\end{itemize}
\subsubsection{Firebase Datenspeicherung}
Die Benutzerdaten der LearnAhead App werden sicher in Firebase gespeichert. Firebase bietet eine Reihe von Funktionen, die die Sicherheit der gespeicherten Daten gewährleisten \cite{firebase_storage}:
\begin{itemize}
    \item \textbf{Datenverschlüsselung:} Firebase speichert Daten in einer verschlüsselten Form, sowohl während der Übertragung als auch im Ruhezustand \cite{firebase_encryption}.
    \item \textbf{Zugriffskontrolle:} Firebase ermöglicht es, detaillierte Sicherheitsregeln festzulegen, um den Zugriff auf die gespeicherten Daten zu kontrollieren \cite{firebase_access_control}.
    \item \textbf{Compliance:} Firebase hält sich an wichtige Compliance-Standards und -Vorschriften, um die Sicherheit und den Schutz der gespeicherten Daten zu gewährleisten \cite{firebase_compliance}.
\end{itemize}
Durch die Verwendung von Firebase konnte die LearnAhead App eine robuste und sichere Lösung für die Authentifizierung und Datenspeicherung implementieren \cite{firebase_overview}.





  \section{Ergebnisse}
\subsection{Der finale Zeitplan}

\begin{table}[H]
  \centering
  \resizebox{\columnwidth}{!}{%
  \begin{tabular}{l|l|l}
  \multicolumn{1}{c|}{\textbf{Meilenstein}} &
    \multicolumn{1}{c|}{\textbf{Zeitplan}} &
    \multicolumn{1}{c}{\textbf{Beschreibung}} \\ \hline
  Literaturrecherche &
    \begin{tabular}[c]{@{}l@{}}14.10.2022 - \\ 31.01.2023\end{tabular} &
    \begin{tabular}[c]{@{}l@{}}Das Durchführen einer umfangreichen Literaturrecherche\\ auf Basis von wissenschaftlichen Dokumenten.\end{tabular} \\ \hline
  Use-Case-Erstellung &
    \begin{tabular}[c]{@{}l@{}}14.10.2022 - \\ 11.11.2022\end{tabular} &
    \begin{tabular}[c]{@{}l@{}}Identifizierung und Dokumentation der \\ Hauptfunktionalitäten und Anwendungsfälle der Lern-App.\end{tabular} \\ \hline
  UI-Konzept &
    \begin{tabular}[c]{@{}l@{}}11.11.2022 - \\ 02.02.2023\end{tabular} &
    \begin{tabular}[c]{@{}l@{}}Entwicklung eines visuellen Konzepts für die \\ Benutzeroberfläche (UI) der Lern-App.\end{tabular} \\ \hline
  Datenbank-Konzept &
    \begin{tabular}[c]{@{}l@{}}20.01.2023 - \\ 16.02.2023\end{tabular} &
    \begin{tabular}[c]{@{}l@{}}Design und Auswahl des Datenbanksystems, die für die \\ App benötigt wird.\end{tabular} \\ \hline
  Architektur-Konzept &
    \begin{tabular}[c]{@{}l@{}}03.02.2023 -\\ 16.02.2023\end{tabular} &
    \begin{tabular}[c]{@{}l@{}}Realisierung einer Code-Architektur und Auswahl  \\ verschiedener Komponenten sowie Framekworks, \\ die in der App verwendet werden.\end{tabular} \\ \hline
  Architektur-Prototyp &
    \begin{tabular}[c]{@{}l@{}}10.02.2023 - \\ 16.02.2023\end{tabular} &
    \begin{tabular}[c]{@{}l@{}}Erstellung eines ersten Prototypen \\  der die vorgeschlagene Architektur implementiert.\end{tabular} \\ \hline
  Login / Registrierung &
    \begin{tabular}[c]{@{}l@{}}17.02.2023 - \\ 30.03.2023\end{tabular} &
    \begin{tabular}[c]{@{}l@{}}Implementierung der Funktionen für Anmeldung, \\ Registrierung und Passwortwiederherstellung.\end{tabular} \\ \hline
  \begin{tabular}[c]{@{}l@{}}Lernkategorien \& \\ Lernziele\end{tabular} &
    \begin{tabular}[c]{@{}l@{}}31.03.2023 -\\ 11.05.2023\end{tabular} &
    \begin{tabular}[c]{@{}l@{}}Implementierung der Funktion zum Erstellen sowie Verwalten\\  von Lernkategorien und -zielen.\end{tabular} \\ \hline
  \begin{tabular}[c]{@{}l@{}}Erstellung von Fragen \\ und Tests\end{tabular} &
    \begin{tabular}[c]{@{}l@{}}12.05.2023 -\\ 08.06.2023\end{tabular} &
    \begin{tabular}[c]{@{}l@{}}Implementierung der Funktion zum Erstellen sowie Verwalten\\  von Fragen und Tests.\end{tabular} \\ \hline
  \begin{tabular}[c]{@{}l@{}}Erstellung von \\ Zusammenfassungen\end{tabular} &
    \begin{tabular}[c]{@{}l@{}}12.05.2023 -\\ 02.06.2023\end{tabular} &
    \begin{tabular}[c]{@{}l@{}}Implementierung der Funktion zum Erstellen sowie  \\ Verwalten von Zusammenfassungen von Lernkategorien.\end{tabular} \\ \hline
  \begin{tabular}[c]{@{}l@{}}Optimale Pausenberechnung \\ realisieren\end{tabular} &
    \begin{tabular}[c]{@{}l@{}}08.06.2023 - \\ 14.06.2023\end{tabular} &
    \begin{tabular}[c]{@{}l@{}}Erstellung eines Algorithmus, welcher den Nutzer die  \\ optimale Pause vorschlägt sowie errinert.\end{tabular} \\ \hline
  \begin{tabular}[c]{@{}l@{}}Optimale Lernplan \\ generieren\end{tabular} &
    \begin{tabular}[c]{@{}l@{}}08.06.2023 -\\ 14.06.2023\end{tabular} &
    \begin{tabular}[c]{@{}l@{}}Erstellung eines optimalen Lernplans auf Basis der Lernziele.\\\end{tabular} \\ \hline
  Durchführen von Tests &
    \begin{tabular}[c]{@{}l@{}}20.06.2023 -\\ 30.06.2023\end{tabular} &
    \begin{tabular}[c]{@{}l@{}}Durchführung von umfassenden Tests, um die Qualität, \\ Funktionalität und Stabilität der App sicherzustellen.\end{tabular} \\ \hline
  Bugs beheben &
    \begin{tabular}[c]{@{}l@{}}01.07.2023 - \\ 16.07.2023\end{tabular} &
    \begin{tabular}[c]{@{}l@{}}Behebung von Fehlern und Problemen in der App.\end{tabular} \\ \hline
  Dokumentation &
    \begin{tabular}[c]{@{}l@{}}14.10.2022 - \\ 16.07.2023\end{tabular} &
    \begin{tabular}[c]{@{}l@{}}Erstellung einer wissenschaftlichen Arbeit, die das Vorgehen, \\ Funktionen, die Implementierung sowie die Verwendung \\ der App begründet.\end{tabular}
  \end{tabular}%
  }
  \end{table}
  \section{User-Tests}
Um Fehlverhalten und mögliche Verständnisprobleme beim Benutzen der App aufzudecken, wurden User Tests durchgeführt. Hierbei wird explizit darauf geachtet, dass nicht nur frei getestet wird, sondern die Tester anhand definierter Aufgaben und Ziele alle Funktionalitäten durchlaufen. \newline
Die User Tests werden deshalb in:
\begin{enumerate}
    \item Offene Tests, allgemeine Ziele
    \item Freie Tests
\end{enumerate}
unterteilt.\\ Im Folgenden findet sich der ausführliche Testbogen, welcher an die Test-User verschickt wurde.\\\\
\textit{
    Danke, dass du dich entschieden hast uns zu helfen und mit uns unsere App zu testen.
    Höchstwahrscheinlich wurdest du von einem von uns beiden angesprochen und gebeten zu helfen.
    Deswegen kann es sein, dass du nicht beide von uns kennst, wir sind: Phillipp Patzelt und Nico Bayer.
    Alle samt Studenten im 6. Semester an der DHBW Ravensburg.\\\\
    Zusammen haben wir im letzten dreiviertel Jahr eine Lern-App erdacht, geplant und erstellt im Rahmen wissenschaftlichen Arbeit.
    Nun sind wir im Endspurt und stehen kurz vor dem offiziellen Release von LearnAhead, einer mobilen Lern-Applikation.\\\\
    Doch nun kurz vor der Veröffentlichung unserer App müssen wir ein letztes mal testen ob auch alles klappt. Hier kommst du ins spiel.
    Bei diesem letzten Test benötigen wir Hilfe. Es kann sein, dass uns durch das montaelange Arbeiten an unserer App, grobe Fehler gar nicht mehr auffallen oder Teile unserer App unpraktisch oder nicht intuitiv zu bedienen sind.\\\\
    Deswegen wurdest du ausgewählt vor dem regulären Betrieb unsere App zu testen und währenddessen diesen Bericht auszufüllen. Bitte gib uns diesen Bogen ausgefüllt bis zum 30.06.2023 zurück, damit wir püntklich vor der Abgabe und dem Release von LearnAhead am 17.07.2023 die letzten Fehler korrigieren können.\\\\
    Bitte scheue dich nicht uns ein ehrliches Feedback zu geben und wir wünschen im Namen von uns beiden viel Spaß und Erfolg bei diesen Tests!
}
\subsection{Offene Tests, allgemeine Ziele}
Das Ziel der offenen Tests ist es, durch eine bewusst grob gewählte Aufgabenstellung, die App auf Praktibilität zu testen, zu sehen ob der Nutzer die App etwas stören könnte und wie intuitiv er sich in der App bewegt.\newline
\textbf{Aufgabe:}\newline
Bitte rufe unsere App auf deinem Mobilgerät auf, falls du sie noch nicht installiert hast, tue dies, indem du im Play Store nach \gqq{LearnAhead} suchst.\\
Du wirst nun auf der Login-Seite landen. Deine erste Aufgabe ist nun auf eigene Faust Folgendes zu tun: \textbf{Eine Zusammenfassung zu erstellen}.\\Versuche von selbst die dafür notwendigen Schritte zu erschließen und auszuführen.\\
Bitte notiere dafür im Folgenden dein Mobilgerät, sowie alles was dir auffällt. Konntest du die Aufgabe leicht ausführen? Gabs es irgendwelche Fehler oder kamst du an irgendeinem Punkt an, an dem du nicht mehr weiterwusstest?\\
Notiere dazu bitte nur konrekt die Antwort auf diese Fragen. Gib bitte erst im nächsten Abschnitt dann dein Feedback oder die Dinge, welche dir in deinen Augen als unpraktisch auffallen.\\
Wenn du deine Aufgabe geschafft hast, verbleibe bitte im Edit-Fenster der Zusammenfassung.\\
Genutztes Mobilgerät: \rule{0.25\textwidth}{0.4pt}\\\\
\textbf{Antwort auf die obige Frage:}\\
\noindent\rule{\textwidth}{0.4pt}\\
\noindent\rule{\textwidth}{0.4pt}\\
\noindent\rule{\textwidth}{0.4pt}\\
\noindent\rule{\textwidth}{0.4pt}\\
\noindent\rule{\textwidth}{0.4pt}\\
\noindent\rule{\textwidth}{0.4pt}\\
\noindent\rule{\textwidth}{0.4pt}\\
\noindent\rule{\textwidth}{0.4pt}\\
\textbf{Feedback zum ersten Eindruck, Wünsche und Dinge die dir unpraktisch vorkommen:}
\noindent\rule{\textwidth}{0.4pt}\\
\noindent\rule{\textwidth}{0.4pt}\\
\noindent\rule{\textwidth}{0.4pt}\\
\noindent\rule{\textwidth}{0.4pt}\\
\noindent\rule{\textwidth}{0.4pt}\\
\noindent\rule{\textwidth}{0.4pt}\\\\
Sehr schön, nun befindest du dich in einem Kernteil unserer App, den Zusammenfassungen. Deine nächste Aufgabe ist es: \textbf{Einen kleinen Markdown Text verfassen und ihn dir anzeigen lassen}.\\
Falls du mit Markdown noch nicht vertraut bist, kein Problem! Im Internet findest du viele Anleitungen wie Markdown funktioniert, hier ein Beispiel: https://www.markdownguide.org/ \\
Folge hierbei der gleichen Prozedur wie vorhin, suche dir selbst einen Weg durch die App. Fallen dir Fehler auf? Wusstest du wo du suchen sollst um die Aufgabe zu erledigen? Im Folgenden darfst du alles was dir auffällt UND dein Feedback und Kritik auf einmal aufschreiben.\\
\noindent\rule{\textwidth}{0.4pt}\\
\noindent\rule{\textwidth}{0.4pt}\\
\noindent\rule{\textwidth}{0.4pt}\\
\noindent\rule{\textwidth}{0.4pt}\\
\noindent\rule{\textwidth}{0.4pt}\\
\noindent\rule{\textwidth}{0.4pt}\\
\noindent\rule{\textwidth}{0.4pt}\\
\noindent\rule{\textwidth}{0.4pt}\\\\
Gut! Nun widmest du dich einem neuen Teil der App und somit auch einer neuen Aufgabe: \textbf{Ändere dein Profilbild}. \\
Wie zuvor, bitte notiere Fehler und Dinge die dir als unpraktisch auffallen.\\
\noindent\rule{\textwidth}{0.4pt}\\
\noindent\rule{\textwidth}{0.4pt}\\
\noindent\rule{\textwidth}{0.4pt}\\
\noindent\rule{\textwidth}{0.4pt}\\
\noindent\rule{\textwidth}{0.4pt}\\
\noindent\rule{\textwidth}{0.4pt}\\
\noindent\rule{\textwidth}{0.4pt}\\\\
Wir sind fast beim letzten Teil der offenen Tests angelangt, nun wird deine Aufgabe sein: \textbf{Ein Lernziel zu erstellen und zu editieren}.\\
Notiere auch hier bitte alle Fehler und alles unpraktische was dir auffällt.\\
\noindent\rule{\textwidth}{0.4pt}\\
\noindent\rule{\textwidth}{0.4pt}\\
\noindent\rule{\textwidth}{0.4pt}\\
\noindent\rule{\textwidth}{0.4pt}\\
\noindent\rule{\textwidth}{0.4pt}\\
\noindent\rule{\textwidth}{0.4pt}\\\\
Beim letzten Teil der offenen Tests geht es darum, \textbf{deine erstellten Elemente zu löschen und dich auszuloggen}.\\
Wie bereits im vorherigen Ablauf notiere alle Fehler und unpraktischen Elemente.\\
\noindent\rule{\textwidth}{0.4pt}\\
\noindent\rule{\textwidth}{0.4pt}\\
\noindent\rule{\textwidth}{0.4pt}\\
\noindent\rule{\textwidth}{0.4pt}\\
\noindent\rule{\textwidth}{0.4pt}\\
\noindent\rule{\textwidth}{0.4pt}\\\\
\subsection{Freie Tests}
In den freien Tests geht es darum, dass du selbst gezielt versuchst etwas kaputt zu machen oder Fehler zu produzieren. Teste bitte auf vielen unterschiedlichen Geräten, ob alles funktioniert und du noch irgendetwas entdeckst, das du bisher noch nicht erwähnt hast. \\
Falls du tatsächlich noch etwas findest, beschreibe uns bitte genau, wie es dazu kam, auf welcher Seite du warst, welches Gerät du genutzt hast und was du geklickt hast.\\
Des Weiteren erwähne hier auch alle noch nicht genannten Verbesserungsvorschläge an unsere App und wir werden sehen, was wir tun können, um diese so schnell wie möglich zu implementieren. Danke und viel Erfolg!\\
\noindent\rule{\textwidth}{0.4pt}\\ % This creates a horizontal line
\noindent\rule{\textwidth}{0.4pt}\\
\noindent\rule{\textwidth}{0.4pt}\\
\noindent\rule{\textwidth}{0.4pt}\\

  \section{Fazit}
LernAhead ist ein Projekt, welches von uns erfolgreich abgeschlossen wurde, jedoch noch einen Weg vor sich hat, der sich erst herauskristallisieren muss.\\
Im folgenden Abschnitt wird der Ablauf des Projekts kritisch betrachtet sowie mögliche zukünftige Features erwägt und daraus ein Resümee gezogen.\\
\subsection{Ablauf des Projekts}
Im Rahmen der Studienarbeit sollen die Studierenden sich \gqq{in ein komplexes, aber eng umgrenztes Gebiet vertiefend einarbeiten und den allgemeinen Stand des Wissens erwerben.}\cite{DHBW_Studienarbeit}\\
Am Beginn von LearnAhead wurden uns von der zuständigen Betreuerin Frau Claudia Zinser zwei wissenschaftliche Bücher ausgehändigt, aus welchen wir Informationen über neuronalen Lernens und andere dem Lernen zugeschriebene Gebiete erfassten.\\
Diese Masse an Wissen musste erst einmal neben dem eigentlichen Hochschulbetrieb verarbeitet werden. Dies zog sich bis in die Mitte der 5. Praxisphase, da nicht nur die beiden ausgehändigten Bücher gelesen wurden, sondern auch andere Dokumente Teil unserer Literaturrecherche waren.\\
Bei der kompletten Durchführung der Studienarbeit war es so, dass diese immer einen kleinen Zeitslot benötigte, egal wann, da es sich hierbei ja um ein kontinuierliches Projekt handelte.\\
Es war interessant und äußerst lehrreich ein Projekt zu erleben, welches von Grundauf, von Anfang bis Ende geplant und implementiert wurde.\\
Wir sind mit dem Ablauf der Studienarbeit sehr zufrieden und haben es eingesehen, dass wir an manchen Stellen bei der Planung überambitioniert waren, jedoch freuen wir uns, das Projekt an kommende Generationen von Studenten weiterzugeben und das Projekt wachsen zu sehen.\\
LearnAhead war kein Projekt, was mit bereits vorhandenem Wissen aus dem Boden gestampft wurde, es wurden uns an vielen Stellen über den Prozess des Lernens, nach nun mehr als fast 15 Jahren Lernen in allen möglichen Schulen, die Augen geöffnet.\\ 
Abseits der Hochschule haben wir nun ein umfassendes Interesse an dem neurologischen Prozess des Lernens, da sich hierfür auch sehr viele Erkenntnisse über maschinelles Lernen ziehen lassen. \\
\subsection{Ausblick und die Zukunft}\label{sec:Ausblick}
Mit der Veröffentlichung von LearnAhead geht ein kleines Kapitel in unserer Geschichte zu Ende, aber ein anderes Tor wird noch viel weiter aufgestoßen, welches in die Zukunft von LearnAhead führt.\\
Viele Features könnten noch in LearnAhead implementiert werden, im nachfolgenden Abschnitt werden einige davon aufgelistet.\\
\begin{enumerate}
    \item Lerninhalte, sowie dessen Tests und Fragen müssen momentan noch manuell eingegeben werden. Dies hat zwar den Vorteil, dass hierdurch die Verbindungen im Gehirn des Lernenden gefestigt werden.\\
    Für die Zukunft gäbe es noch die Möglichkeit die App in so einer Form zu erweitern, dass Lerninhalte und Tests oder ganze Lernkategorien importiert, beziehungsweise exportiert werden können, so dass ein neuer Jahrgang den vergangen Jahrgang nach der Lernkategorie für die anstehende Klausur fragen könnte.
    \item Ein anderes Team aus Studierenden haben im Rahmen ihrer Studienarbeit eine Lern-App entwickelt, welche ihren Fokus auf kollaboratives Lernen legt. In Absprache mit der zuständigen Betreuerin Claudia Zinser könnten hierbei beide Lern-Apps über eine API oder Ähnliches miteinander verbunden werden.
    \item Bei den Zusammenfassungen ist es aktuell noch mühsam die Inhalte selber von Grund auf einzugeben. Um dem entgegenzuwirken könnte ein Template-System aufgesetzt werden, welches es den Nutzern ermöglicht Vorlagen für Zusammenfassungen zu erstellen.
    \item Bisher wird dem User das Lernen nur durch einige zufällig ausgewählte Lerntipps beigebracht. Unsere Vision am Anfang von LearnAhead war ein Bereich der App, in welchem man das Lernen lernen kann und allgemeine Falschannahmen zum lernen aufgeklärt werden.
    \item Der User wird dafür belohnt, wenn er eine Frage nach mehrmaligem falsch beantworten richtig beantwortet. Es könnte zur Unterstützung des Lernprozesses nun ein Algorithmus implementiert werden, welcher den User nach mehrmaligem falsch beantworten einer Frage auf die zugehörige Zusammenfassung hinleitet.\\
    Hierfür müssten Zusammenfassungsinhalte mit Fragen verknüpft werden. Dies könnte beispielsweise über die Tag-Funktion geschehen, indem man einen Tag die Zugehörigkeit zu einer Zusammenfassung zuspricht.
    \item Der Markdown-Editor unterstützt bisher nur Basic-Markdown Syntax. Markwon stellt die Möglichkeit zur Verfügung Plugins einzubauen, welche diese Syntax erweitern. Es könnte eine Funktion zum einfügen von mathematischen Operationen implementiert werden, wie es MathJax (\href{https://www.mathjax.org/}{https://www.mathjax.org/}) in LaTeX implementiert.
    \item Momentan wird der Lernfortschritt einer Lernkategorie dem Benutzer nicht vollständig ersichtlich. In der weiteren Entwicklung des Projektes könnte deswegen ein Lernfortschritts-Balken implementiert werden, welcher dem User seinen momentan Lernfortschritt einer Lernkategorie anzeigt.
    \item Tags sind momentan statisch anlegbar. Dies bedeutet, dass wenn ein Tag erstellt wird und dabei ein Fehler unterläuft, muss dieser gelöscht und korrekt neu erstellt werden. Um dem entgegenzuwirken könnte man ein Tag-Bearbeitungssystem entwickeln.
    \item LearnAhead ist momentan durch seine Plattform-Abhängigkeit nur auf Android Mobilgeräte beschränkt. Um dies zu verbessern könnte LearnAhead zu einer PWA umfunktioniert werden, um Plattformunabhängigkeit zu ermöglichen.
    \item Derzeit muss der User sich die Daten für seine Lernziele quantitativ merken und selbst verwalten. Dieses Problem könnte mit einem Kalender-System gelöst werden, welches dem User eine bessere zeitliche Übersicht über seine Lernziele gibt.
    \item Wir haben uns entschieden, unsere App nicht im Google Play Store zu veröffentlichen, da dies im Rahmen unseres Studiums nicht erforderlich ist. Eine der Herausforderungen bei der Veröffentlichung einer App im Play Store ist die Notwendigkeit einer umfassenden Datenschutzerklärung, sowie das dreimalige nicht akzeptieren von gültigen amtlichen Ausweisdokumenten seitens Google. Diese zu erstellen und sicherzustellen, dass unsere App vollständig konform ist, würde zusätzliche Zeit und Ressourcen erfordern. Da unser Hauptziel darin besteht, die technischen sowie wissenschaftlichen Fähigkeiten und Kenntnisse zu erwerben, die für die Entwicklung einer funktionsfähigen App erforderlich sind, haben wir uns darauf konzentriert, eine App zu erstellen, die unseren Anforderungen entspricht, anstatt sie für eine breitere Öffentlichkeit zugänglich zu machen. Um die App für die breite Öffentlichkeit zugänglich zu machen, könnte man diesen Schritt noch in der Zukunft von LearnAhead gehen.

\end{enumerate}

  %%%%%%%%%%%%%%%%%%%%%%% Literaturverzeichnis %%%%%%%%%%%%%%%%%%%%%%%
  \printbibliography[title=Referenzen]
\addcontentsline{toc}{section}{Referenzen}
\newpage


  %%%%%%%%%%%%%%%%%%%%%%%%%%%%%% Anhang %%%%%%%%%%%%%%%%%%%%%%%%%%%%%%
  \renewcommand{\thetable}{\Alph{section}.\arabic{table}}
  \renewcommand{\thefigure}{\Alph{section}.\arabic{figure}}
  \renewcommand{\thelstlisting}{\Alph{section}.\arabic{lstlisting}}
  \pagenumbering{Alph}

  % \include{pages/anhang}
\end{document}